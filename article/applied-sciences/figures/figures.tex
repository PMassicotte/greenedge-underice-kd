\documentclass[12pt,a4paper]{scrartcl}
\usepackage[utf8]{inputenc}
\usepackage{amsmath}
\usepackage{amsfonts}
\usepackage{amssymb}
\usepackage{graphicx}

\usepackage[bottom = 1in, left = 0.5in, right = 0.5in, top = 1in]{geometry}

\usepackage[english]{babel}
\usepackage[autostyle]{csquotes}
\usepackage{mathptmx}

\usepackage[labelfont=bf]{caption}

\usepackage[default, scale=0.95]{opensans}

\usepackage[T1]{fontenc}

\usepackage{fixltx2e}

\newcommand{\ked}{\ensuremath{K_{Ed}}}
\newcommand{\klu}{\ensuremath{K_{Lu}}}
\newcommand{\edz}{\ensuremath{{E_d(z)}}}
\newcommand{\luz}{\ensuremath{{L_u(z)}}}
\newcommand{\edzero}{\ensuremath{{E_d(0^-)}}}
\newcommand{\meanedz}{\ensuremath{{\overline{E_d}(z)}}}
\newcommand{\meanluz}{\ensuremath{{\overline{L_u}(z)}}}

% \addto\captionsenglish{\renewcommand{\figurename}{Fig.}}
% \addto\captionsenglish{\renewcommand{\tablename}{Supplementary Table}}

\title{Figures}
\date{}

\begin{document}
\maketitle

\begin{figure}[ht]
	\centering
	\includegraphics[scale = 1]{../../../graphs/fig1.pdf}
	\caption{Spatial configuration used for the 3D Monte-Carlo numerical simulation. Percentage covered by the melt pond over the area described by the black lines are also presented. For each of these areas, light profiles were averaged (see Fig. 6). For visualization, lines of the sampling distances have been plotted only at 10 meters intervals.}
\end{figure}

\clearpage
\newpage

\begin{figure}[ht]
	\centering
	\includegraphics[scale = 1]{../../../graphs/fig2.pdf}
	\caption{Examples of downwelling irradiance (\edz{}) and upwelling radiance (\luz{}) light profiles taken under-ice on 2016-06-20. Note the subsurface light maximum for the irradiance profiles that are not present in the radiance profiles.}
\end{figure}

\clearpage
\newpage

\begin{figure}[ht]
	\centering
	\includegraphics[scale = 1]{../../../graphs/fig3.pdf}
	\caption{Comparison of downwelling irradiance (\edz{}) and upwelling radiance (\luz{}) of one light profile taken under-ice. Profiles were normalized to light at 10 meters (under subsurface light maximum) in order to emphasize the similar shape between \edz{} and \luz{}.}
\end{figure}

\clearpage
\newpage

\begin{figure}[ht]
	\centering
	\includegraphics[scale = 1]{../../../graphs/fig4.pdf}
	\caption{Scatter plots showing the relationships between downwelling irradiance (\edz{}) and upwelling radiance (\luz{}) between 400 and 580 nm at different depths (numbers in gray boxes). Blue lines and shaded areas represent respectively the regressions lines and the confidence intervals of the fitted linear models. Dotted lines are the 1:1 lines.}
\end{figure}

\clearpage
\newpage

\begin{figure}[ht]
	\centering
	\includegraphics[scale = 1]{../../../graphs/fig5.pdf}
	\caption{Cross-sections of simulated irradiance and radiance light profiles going through the 5-meters radius melt pond. The logarithm of normalized number of photons to the value observed at 0.5 meters depth at 50 meters from the center of the melt pond has been used for visualization.}
\end{figure}

\clearpage
\newpage

\begin{figure}[ht]
	\centering
	\includegraphics[scale = 1]{../../../graphs/fig6.pdf}
	\caption{Simulated irradiance and radiance light profiles (black lines, $n$ = 403) along with averaged light profiles (colored lines, \meanedz{}, \meanluz{}) from five different areas with varying proportions of the surface occupied by the melt pond (see Fig. 1). Note that some of the irradiance light profiles show the same subsurface light maxima observed with in-situ data (see Fig. 2).}
\end{figure}

\clearpage
\newpage

\begin{figure}[ht]
	\centering
	\includegraphics[scale = 1]{../../../graphs/fig7.pdf}
	\caption{Simulated irradiance and radiance lights profiles at difference distances from the center of the melt pond (see Fig. 1) used to compute \ked{} and \klu{}. These attenuation coefficients were used to propagate surface reference light, \edzero{}, surface values of the colored lines in Fig. 6) through the water column.}
\end{figure}

\clearpage
\newpage

\begin{figure}[ht]
	\centering
	\includegraphics[scale = 1]{../../../graphs/fig8.pdf}
	\caption{Reference irradiance profiles (black lines, see Fig. 6) and propagated light using \ked{} and \klu{} (see Fig. 7). Propagated light was done using reference irradiance at 0.5 m from the surface.}
\end{figure}

\clearpage
\newpage

\begin{figure}[ht]
	\centering
	\includegraphics[scale = 1]{../../../graphs/fig9.pdf}
	\caption{Relative errors of the predictions calculated as the relative differences between the integral of the reference and predicted light profiles. Note that the horizontal distances are reported as the distances from the ice ridge outside the melt pond.}
\end{figure}

\end{document}
