\documentclass[12pt,a4paper]{scrartcl}
    \usepackage[utf8]{inputenc}
    \usepackage{amsmath}
    \usepackage{amsfonts}
    \usepackage{amssymb}
    \usepackage{graphicx}

    \usepackage[bottom = 1in, left = 0.5in, right = 0.5in, top = 1in]{geometry}

    \usepackage[english]{babel}
    \usepackage[autostyle]{csquotes}
    \usepackage{mathptmx}

    \usepackage[labelfont=bf]{caption}

    \usepackage[default, scale=0.95]{opensans}

    \usepackage[T1]{fontenc}

    \usepackage{fixltx2e}

    % \addto\captionsenglish{\renewcommand{\figurename}{Fig.}}
    % \addto\captionsenglish{\renewcommand{\tablename}{Supplementary Table}}

    \title{Figures}
    \date{}

    \begin{document}
    \maketitle

     \begin{figure}[ht]
         \centering
         \includegraphics[scale = 1]{../../../graphs/fig1.pdf}
         \caption{Example of downwelling irradiance (Ed) and upwelling radiance (Lu) light profiles taken under-ice (between 400 and 700 nm). Note the subsurface light maximum for the irradiance profiles that are not present in the radiance profiles.}
     \end{figure}

    \clearpage
    \newpage

    \begin{figure}[ht]
        \centering
        \includegraphics[scale = 1]{../../../graphs/fig2.pdf}
        \caption{Comparison of downwelling irradiance (Ed) and upwelling radiance (Lu) of one light profile taken under-ice. Profiles were normalized to light at 10 meters (under subsurface light maximum) in order to emphasize the similar shape between Ed and Lu.}
    \end{figure}

    \clearpage
    \newpage

    \begin{figure}[ht]
        \centering
        \includegraphics[scale = 1]{../../../graphs/fig3.pdf}
        \caption{Scatter plots showing the relationships between downwelling irradiance (Ed) and upwelling radiance (Lu). Blue lines and shaded areas represent respectively the regressions lines and the confidence intervals of the fitted linear models. Dotted lines are the 1:1 lines.}
    \end{figure}

    \clearpage
    \newpage

    \begin{figure}[ht]
        \centering
        %\includegraphics[scale = 1]{../../../graphs/fig4.pdf}
        \caption{SimulO geometry.}
    \end{figure}

    \clearpage
    \newpage

    \begin{figure}[ht]
        \centering
        \includegraphics[scale = 1]{../../../graphs/fig5.pdf}
        \caption{Crossections of irradiance and radiance light profiles simulated with SimulO. For visualization, the number of photons have been normalized between 0 and 1.}
    \end{figure}

    \clearpage
    \newpage

    \begin{figure}[ht]
        \centering
        \includegraphics[scale = 1]{../../../graphs/fig6.pdf}
        \caption{Averaged irradiance and radiance reference light profiles from five different classes of distances from the center of the melt pond. The proportion of the surface occupied by the melt pond (5 meters radius) is presented in parenthesis.}
    \end{figure}

    \clearpage
    \newpage

    \begin{figure}[ht]
        \centering
        \includegraphics[scale = 1]{../../../graphs/fig7.pdf}
        \caption{Simulated irradiance and radiance lights profiles at difference distances from the center of the melt pond used to compute KEd and KLu.}
    \end{figure}

    \clearpage
    \newpage

    \begin{figure}[ht]
        \centering
        \includegraphics[scale = 1]{../../../graphs/fig8.pdf}
        \caption{Reference irradiance profiles (black lines, see Fig. 6) and propagated light using KEd and KLu (see Fig. 7). Propagated light was done using reference irradiance at 0.5 mfrom the surface.}
    \end{figure}

    \clearpage
    \newpage

    \begin{figure}[ht]
        \centering
        \includegraphics[scale = 1]{../../../graphs/fig9.pdf}
        \caption{Relative errors of the predictions calculated as the relative differences between the integral of the reference and predicted light profiles.}
    \end{figure}

    \clearpage
    \newpage

    \begin{figure}[ht]
        \centering
        %\includegraphics[scale = 1]{../../../graphs/fig9.pdf}
        \caption{Methody figure (reprojecting Ed from KLu using insitu data).}
    \end{figure}

\end{document}
