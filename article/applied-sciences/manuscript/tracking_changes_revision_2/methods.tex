\section{Material and methods}

\subsection{Study site and field campaign}

The field campaign was part of the GreenEdge project (www.greenedgeproject.info) which was conducted on landfast ice southeast of the Qikiqtarjuaq Island in the Baffin Bay (67.4797N, 63.7895W). The field operations took place at an ice camp where the water depth was 360 m, from April 20 to July 27, 2016 (Supplementary Fig. \DIFdelbegin \DIFdel{1}\DIFdelend \DIFaddbegin \DIFadd{A1}\DIFaddend ). During the sampling period, the study site experienced changes in the snow cover and landfast ice thickness of \DIFdelbegin \DIFdel{between }\DIFdelend 0-49 \DIFaddbegin \DIFadd{cm }\DIFaddend and 106-149 cm, respectively.

\subsection{In situ underwater light measurements}

During the campaign, a total of 83 vertical light profiles were acquired using a factory-calibrated ICE-Pro (an ice floe version of the C-OPS, or Compact-Optical Profiling System, from Biospherical Instruments Inc.) equipped with both downward plane irradiance \edz{} (\wmsquare{}) and upward radiance \luz{} (\wmsquaresr{}) radiometers. The ICE-Pro system is a negatively buoyant instrument with a cylindrical shape 10 inches in diameter and is not designed for free-fall casts (as opposed to its open-water version). To perform the profiles, the frame was manually lowered into an auger hole that had been cleaned of ice chunks. Once it was underneath the ice layer, fresh clean snow was shovelled back in the hole to prevent the creation of a bright spot right on top of the sensors. Great care was taken not to pollute the hole surroundings (footsteps, water and slush spillage from the auger drilling, etc.). The operator then stepped back 50 m, while keeping the sensors right under the ice, to avoid any human shadow on top of the profile. The frame was then lowered manually at a constant descent rate of approximately 0.3 m s\textsuperscript{-1}. The above-surface atmospheric reference sensor was fixed on a steady tripod standing on the floe approximately 2 m above the surface and above all neighbouring ice camp features. Data processing and validation were performed using a protocol inspired by the one proposed by \citet{Smith1984}\DIFdelbegin \DIFdel{which is now used by the main space agencies}\DIFdelend . Measurements were made at 19 wavelengths: 380, 395, 412, 443, 465, 490, 510, 532, 555, 560, 589, 625, 665, 683, 694, 710, 765, 780 and 875 nm. For this study, \ed{} and \lu{} spectra were interpolated linearly between 400 and 700 nm every 10 nm. In situ diffuse attenuation coefficients ($K$) for both \ed{} (\ked{}) and \lu{} (\klu{}) were calculated on a 5 m sliding window (10--15 m, 15--20 m, $\ldots$, 70--75 m, 75--80 m) starting at 10 m \DIFaddbegin \DIFadd{depth }\DIFaddend to reduce the effects of surface heterogeneity. A total of 72 044 non-linear models were calculated to estimate both $K$ coefficients from Equation 1 (83 profiles $\times$ 14 depths $\times$ 31 wavelengths $\times$ 2 radiometric quantities (\ed{}, \lu{})). A conservative \rsquared{} of 0.99 was used essentially to filter out noisy profiles. 42 407 models were kept for subsequent analysis.

\subsection{3D Monte Carlo numerical simulations \DIFaddbegin \DIFadd{of radiative transfer}\DIFaddend }

\subsubsection{Theory and geometry}

3D numerical Monte Carlo simulation is a convenient approach for modelling the light field under spatially heterogeneous sea surfaces \citep{Mobley_ocean_optics_book, Petrich2012, Katlein2014, Katlein2016}. They are simple to understand and versatile, and incident light, IOPs and geometry can be easily changed. In this study, we used SimulO, a 3D Monte Carlo software program that simulates the propagation of light in optical instruments or in ocean waters \citep{Leymarie2010}. Our objective was to simulate the propagation of sunlight underneath heterogeneous ice-covered ocean waters. Simulations were performed in an idealized ocean described by a cylinder of 120 m radius and 150 m depth (Fig. 1). The water IOPs were selected to reflect pre-bloom conditions in the green/blue spectral region (\DIFdelbegin \DIFdel{a = b = 0.05 }\DIFdelend \DIFaddbegin \DIFadd{$a = b = 0.05$ }\DIFaddend \mminus{}). These typical averaged values were measured during the GreenEdge 2016 campaign using an in situ spectrophotometer (ac-s from Sea-Bird Scientific) and represent the contribution of both pure water and the water constituents. The scattering phase function was described by a Fournier-Forand analytic form with a 3\% backscatter fraction \citep{Fournier1994, Mobley2002}. The inclusion of a 3D sea ice layer at the upper boundary of the ocean would require extensive computing power because of the high scattering properties of sea ice. Instead, sea ice was incorporated at the upper boundary of the ocean using a 2D light-emitting surface with a radius of 100 m. The angular distribution and \DIFdelbegin \DIFdel{amplitude }\DIFdelend \DIFaddbegin \DIFadd{magnitude }\DIFaddend of the light field emitted by the surface was chosen to mimic observed field data \citep{Girard2018}. SimulO does not allow the use of arbitrary angular distribution for photon-emitting surfaces. To overcome this problem, two sources of photons were summed up in order to reproduce an observed under-ice light field (Fig. 2). The first source was a regular Lambertian emitting surface while the second was a Lambertian emitting surface but restricted to an emission within 60 degrees of the zenith angle. A 5-m radius melt pond was set up at the centre of the emitting surface (Fig. 1). The melt pond had the same emitting angular distribution as the surrounding ice. Its intensity was four times higher than the surrounding ice, which corresponds to typical conditions found in the Arctic during summer \citep{Perovich2016}.

Given our interest in surface light profiles, 2D horizontal software detectors were placed vertically every 0.5 m, from 0.5 m up to a depth of 25 m. Detectors include 1 m\textsuperscript{2} pixels measuring downward irradiance and upward radiance (\DIFdelbegin \DIFdel{5 degree half angle }\DIFdelend \DIFaddbegin \DIFadd{5-degree half angle of acceptance}\DIFaddend ). In order to avoid the effect of the boundary (i.e.\DIFaddbegin \DIFadd{, }\DIFaddend absorption by the side of the cylinder used to simulate the water column), data outside a radius of 50 m were not used (see the green box in Fig. 1). A total number of $7.14 \times 10^{10}$ photons were simulated to obtain a sufficient number of upwelling photons. The simulation took approximately 6 000 hours distributed over 2 000 CPU cores. \DIFdelbegin \DIFdel{Since }\DIFdelend \DIFaddbegin \DIFadd{Because }\DIFaddend the geometry was symmetrical azimuthally, irradiance and radiance were averaged over the azimuth in order to \DIFdelbegin \DIFdel{raise }\DIFdelend \DIFaddbegin \DIFadd{increase }\DIFaddend the signal-to-noise ratio. \DIFdelbegin \DIFdel{Due to }\DIFdelend \DIFaddbegin \DIFadd{Because of }\DIFaddend the low scattering coefficients used to reproduce in situ conditions observed during the sampling campaign, radiance profiles were noisy because a small number of upward photons could be captured. To address this issue, radiance profiles were smoothed using a Gaussian fit (Supplementary Fig. \DIFdelbegin \DIFdel{2}\DIFdelend \DIFaddbegin \DIFadd{A2}\DIFaddend ). 

\subsubsection{Estimation of reference and local light profiles}

To explore how the melt pond influences the averaged underwater irradiance and radiance profiles (Fig. 1), data from the Monte Carlo simulation were averaged according to six different radii, corresponding to varying melt pond spatial proportions. The simulated light profiles were averaged within the following surface areas: (1) 10 m radius (25\% melt pond cover), (2) 11.18 m radius (20\% melt pond cover), (3) 12.91 m radius (15\% melt pond cover), (4) 15.81 m radius (10\% melt pond cover), (5) 22.36 m radius (5\% melt pond cover) and (6) 50 m radius (1\% melt pond cover). For each of these six configurations, the corresponding averaged light profile, \meanedz{}, was subsequently viewed as an adequate description of the average underwater light field. For the remainder of the text, these averaged profiles are referred to as reference light profiles. Furthermore, 50 light profiles, evenly spaced by 1 m from the melt pond centre, were extracted to mimic local measurements of light and to calculate associated diffuse attenuation coefficients.

\subsection{Statistical analysis}

All statistical analyses and graphics were carried out with R 3.5.1 \citep{RCoreTeam2018}. 