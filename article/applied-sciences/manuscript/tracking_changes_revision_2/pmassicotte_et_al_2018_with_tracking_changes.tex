%  LaTeX support: latex@mdpi.com 
%DIF LATEXDIFF DIFFERENCE FILE
%DIF DEL ./revision_1/pmassicotte_et_al_2018.tex   Wed Nov  7 07:56:47 2018
%DIF ADD ./revision_2/pmassicotte_et_al_2018.tex   Tue Nov 13 09:56:22 2018
%  In case you need support, please attach all files that are necessary for compiling as well as the log file, and specify the details of your LaTeX setup (which operating system and LaTeX version / tools you are using).

% You need to save the "mdpi.cls" and "mdpi.bst" files into the same folder as this template file.

%=================================================================
\documentclass[applsci,article,submit,moreauthors,pdftex,10pt,a4paper]{Definitions/mdpi} 

% If you would like to post an early version of this manuscript as a preprint, you may use preprint as the journal and change 'submit' to 'accept'. The document class line would be, e.g. \documentclass[preprints,article,accept,moreauthors,pdftex,10pt,a4paper]{mdpi}. This is especially recommended for submission to arXiv, where line numbers should be removed before posting. For preprints.org, the editorial staff will make this change immediately prior to posting.

%
%--------------------
% Class Options:
%--------------------
% journal
%----------
% Choose between the following MDPI journals:
% acoustics, actuators, addictions, admsci, aerospace, agriculture, agronomy, algorithms, animals, antibiotics, antibodies, antioxidants, applsci, arts, asi, atmosphere, atoms, axioms, batteries, bdcc, behavsci, beverages, bioengineering, biology, biomedicines, biomimetics, biomolecules, biosensors, brainsci, buildings, carbon, cancers, catalysts, cells, ceramics, challenges, chemengineering, chemosensors, children, cleantechnol, climate, clockssleep, cmd, coatings, colloids, computation, computers, condensedmatter, cosmetics, cryptography, crystals, cybersecurity, data, dentistry, designs, diagnostics, dairy, diseases, diversity, drones, econometrics, economies, education, electrochem, electrochemistry, electronics, energies, entropy, environments, epigenomes, est, fermentation, fibers, fire, fishes, fluids, foods, forecasting, forests, fractalfract, futureinternet, galaxies, games, gastrointestdisord, gels, genealogy, genes, geohazards, geosciences, geriatrics, hazardousmatters, healthcare, heritage, highthroughput, horticulturae, humanities, hydrology, informatics, information, infrastructures, inorganics, insects, instruments, ijerph, ijfs, ijms, ijgi, ijtpp, inventions, j, jcdd, jcm, jcs, jdb, jfb, jfmk, jimaging, jof, jintelligence, jlpea, jmmp, jmse, jpm, jrfm, jsan, land, languages, laws, life, literature, logistics, lubricants, machines, magnetochemistry, make, marinedrugs, materials, mathematics, mca, medsci, medicina, medicines, membranes, metabolites, metals, microarrays, micromachines, microorganisms, minerals, modelling, molbank, molecules, mps, mti, nanomaterials, ncrna, neonatalscreening, neuroglia, nitrogen, nutrients, ohbm, particles, pathogens, pharmaceuticals, pharmaceutics, pharmacy, philosophies, photonics, plants, plasma, polymers, polysaccharides, proceedings, processes, proteomes, publications, quaternary, qubs, reactions, recycling, religions, remotesensing, reports, resources, risks, robotics, safety, sci, scipharm, sensors, separations, sexes, sinusitis, smartcities, socsci, societies, soilsystems, sports, standards, stats, surfaces, surgeries, sustainability, symmetry, systems, technologies, toxics, toxins, tropicalmed, universe, urbansci, vaccines, vehicles, vetsci, vibration, viruses, vision, water, wem, wevj
%---------
% article
%---------
% The default type of manuscript is article, but can be replaced by: 
% abstract, addendum, article, benchmark, book, bookreview, briefreport, casereport, changes, comment, commentary, communication, conceptpaper, correction, conferenceproceedings, conferencereport, expressionofconcern, meetingreport, creative, datadescriptor, discussion, editorial, essay, erratum, hypothesis, interestingimages, letter, meetingreport, newbookreceived, opinion, obituary, projectreport, reply, reprint, retraction, review, perspective, protocol, shortnote, supfile, technicalnote, viewpoint
% supfile = supplementary materials
% protocol: If you are preparing a "Protocol" paper, please refer to http://www.mdpi.com/journal/mps/instructions for details on its expected structure and content.
%----------
% submit
%----------
% The class option "submit" will be changed to "accept" by the Editorial Office when the paper is accepted. This will only make changes to the frontpage (e.g. the logo of the journal will get visible), the headings, and the copyright information. Also, line numbering will be removed. Journal info and pagination for accepted papers will also be assigned by the Editorial Office.
%------------------
% moreauthors
%------------------
% If there is only one author the class option oneauthor should be used. Otherwise use the class option moreauthors.
%---------
% pdftex
%---------
% The option pdftex is for use with pdfLaTeX. If eps figures are used, remove the option pdftex and use LaTeX and dvi2pdf.

%=================================================================
\firstpage{1} 
\makeatletter 
\setcounter{page}{\@firstpage} 
\makeatother
\pubvolume{xx}
\issuenum{1}
\articlenumber{5}
\pubyear{2018}
\copyrightyear{2018}
%\externaleditor{Academic Editor: name}
\history{Received: date; Accepted: date; Published: date}
%\updates{yes} % If there is an update available, un-comment this line

%% MDPI internal command: uncomment if new journal that already uses continuous page numbers 
%\continuouspages{yes}

%------------------------------------------------------------------
% The following line should be uncommented if the LaTeX file is uploaded to arXiv.org
%\pdfoutput=1

%=================================================================
% Add packages and commands here. The following packages are loaded in our class file: fontenc, calc, indentfirst, fancyhdr, graphicx, lastpage, ifthen, lineno, float, amsmath, setspace, enumitem, mathpazo, booktabs, titlesec, etoolbox, amsthm, hyphenat, natbib, hyperref, footmisc, geometry, caption, url, mdframed, tabto, soul, multirow, microtype, tikz

%=================================================================
%% Please use the following mathematics environments: Theorem, Lemma, Corollary, Proposition, Characterization, Property, Problem, Example, ExamplesandDefinitions, Hypothesis, Remark, Definition
%% For proofs, please use the proof environment (the amsthm package is loaded by the MDPI class).

%=================================================================
% Full title of the paper (Capitalized)
\Title{Estimating underwater light regime under spatially heterogeneous sea ice in the Arctic}

% Author Orchid ID: enter ID or remove command
\newcommand{\orcidauthorA}{0000-0002-5919-4116} % Add \orcidA{} behind the author's name
\newcommand{\orcidauthorB}{0000-0002-2526-0808} % Add \orcidB{} behind the author's name
\newcommand{\orcidauthorC}{0000-0001-9705-407X} % Add \orcidB{} behind the author's name

% Authors, for the paper (add full first names)
\Author{Philippe Massicotte $^{1,*}$\orcidA{}, Guislain Bécu $^{1}$\orcidB{}, Simon Lambert-Girard $^{1}$, Edouard Leymarie $^{2}$\orcidC{} and Marcel Babin $^{1}$}

% Authors, for metadata in PDF
\AuthorNames{Philippe Massicotte, Guislain Bécu, Simon Lambert-Girard, Edouard Leymarie and Marcel Babin}

% Affiliations / Addresses (Add [1] after \address if there is only one affiliation.)
\address{%
$^{1}$ \quad Takuvik Joint International Laboratory (UMI 3376) Université Laval (Canada) Centre National de la Recherche Scientifique (France)\\
$^{2}$ \quad Sorbonne Université, CNRS, Laboratoire d'Océanographie de Villefranche, LOV, F-06230 Villefranche-sur-Mer, France}

% Contact information of the corresponding author
\corres{Correspondence: philippe.massicotte@takuvik.ulaval.ca}

%DIF 90a90-91
\usepackage[final]{pdfpages} %DIF > 
 %DIF > 
%DIF -------
\newcommand{\ked}{\ensuremath{K_{d}}}
\newcommand{\klu}{\ensuremath{K_{Lu}}}
\newcommand{\edz}{\ensuremath{{E_d(z)}}}
\newcommand{\luz}{\ensuremath{{L_u(z)}}}
\newcommand{\ed}{\ensuremath{{E_d}}}
\newcommand{\lu}{\ensuremath{{L_u}}}
\newcommand{\edzero}{\ensuremath{{E_d(0^-)}}}
\newcommand{\meanedz}{\ensuremath{{\overline{E_d}(z)}}}
\newcommand{\meanluz}{\ensuremath{{\overline{L_u}(z)}}}
\newcommand{\meanked}{\ensuremath{{\overline{K_{d}}}}}

%  Units
\newcommand{\mminus}{m\textsuperscript{-1}}
\newcommand{\wmsquare}{W~m\textsuperscript{-2}}
\newcommand{\wmsquaresr}{W~m\textsuperscript{-2}~sr\textsuperscript{-1}}

%DIF 106c108
%DIF < \newcommand{\rsquared}{R\textsuperscript{2}}
%DIF -------
\newcommand{\rsquared}{\ensuremath{R^2}} %DIF > 
%DIF -------


% Current address and/or shared authorship
%\firstnote{Current address: Affiliation 3} 
%\secondnote{These authors contributed equally to this work.}
% The commands \thirdnote{} till \eighthnote{} are available for further notes

%\simplesumm{} % Simple summary

%\conference{} % An extended version of a conference paper

% Abstract (Do not insert blank lines, i.e. \\) 
%DIF 119c121
%DIF < \abstract{The vertical diffuse attenuation coefficient for downward irradiance (\ked{}) is an apparent optical property commonly used in primary production models to propagate incident solar radiation in the water column. In open water, estimating \ked{} is relatively straightforward when a vertical profile of measurements of downward irradiance, \ed{}, is available. In the Arctic, the ice pack is characterized by a complex mosaic composed of sea ice with snow, ridges, melt ponds and leads. Because of the resulting spatially heterogeneous light field in the first metres of the water column, it is difficult to measure at single-point locations meaningful \ked{} values that allow predicting average irradiance at any depth. The main objective of this work is to propose a new method to estimate average irradiance over large spatially heterogeneous area as it would be seen by drifting phytoplankton. Using both in situ data and 3D Monte Carlo numerical simulations, we show that (1) the large-area average downward irradiance profile, \meanedz{}, under heterogeneous sea ice cover can be represented by a single-term exponential function and (2) the vertical attenuation coefficient for upward radiance (\klu{}), which is up to two times less influenced by a heterogeneous incident light field than \ked{} in the vicinity of a melt pond, can be used as a proxy to estimate \meanedz{} in the water column.}
%DIF -------
\abstract{The vertical diffuse attenuation coefficient for downward plane irradiance (\ked{}) is an apparent optical property commonly used in primary production models to propagate incident solar radiation in the water column. In open water, estimating \ked{} is relatively straightforward when a vertical profile of measurements of downward irradiance, \ed{}, is available. In the Arctic, the ice pack is characterized by a complex mosaic composed of sea ice with snow, ridges, melt ponds, and leads. Because of the resulting spatially heterogeneous light field in the top meters of the water column, it is difficult to measure at single-point locations meaningful \ked{} values that allow predicting average irradiance at any depth. The main objective of this work is to propose a new method to estimate average irradiance over large spatially heterogeneous area as it would be seen by drifting phytoplankton. Using both in situ data and 3D Monte Carlo numerical simulations of radiative transfer, we show that (1) the large-area average vertical profile of downward irradiance, \meanedz{}, under heterogeneous sea ice cover can be represented by a single-term exponential function and (2) the vertical attenuation coefficient for upward radiance (\klu{}), which is up to two times less influenced by a heterogeneous incident light field than \ked{} in the vicinity of a melt pond, can be used as a proxy to estimate \meanedz{} in the water column.} %DIF > 
%DIF -------

% Keywords
\keyword{apparent optical properties, 3D Monte Carlo numerical simulations, downward irradiance, upward radiance, sea ice heterogeneity, vertical attenuation coefficient, melt ponds}

%\setcounter{secnumdepth}{4}
%%%%%%%%%%%%%%%%%%%%%%%%%%%%%%%%%%%%%%%%%%
%DIF PREAMBLE EXTENSION ADDED BY LATEXDIFF
%DIF UNDERLINE PREAMBLE %DIF PREAMBLE
\RequirePackage[normalem]{ulem} %DIF PREAMBLE
\RequirePackage{color}\definecolor{RED}{rgb}{1,0,0}\definecolor{BLUE}{rgb}{0,0,1} %DIF PREAMBLE
\providecommand{\DIFadd}[1]{{\protect\color{blue}\uwave{#1}}} %DIF PREAMBLE
\providecommand{\DIFdel}[1]{{\protect\color{red}\sout{#1}}}                      %DIF PREAMBLE
%DIF SAFE PREAMBLE %DIF PREAMBLE
\providecommand{\DIFaddbegin}{} %DIF PREAMBLE
\providecommand{\DIFaddend}{} %DIF PREAMBLE
\providecommand{\DIFdelbegin}{} %DIF PREAMBLE
\providecommand{\DIFdelend}{} %DIF PREAMBLE
%DIF FLOATSAFE PREAMBLE %DIF PREAMBLE
\providecommand{\DIFaddFL}[1]{\DIFadd{#1}} %DIF PREAMBLE
\providecommand{\DIFdelFL}[1]{\DIFdel{#1}} %DIF PREAMBLE
\providecommand{\DIFaddbeginFL}{} %DIF PREAMBLE
\providecommand{\DIFaddendFL}{} %DIF PREAMBLE
\providecommand{\DIFdelbeginFL}{} %DIF PREAMBLE
\providecommand{\DIFdelendFL}{} %DIF PREAMBLE
%DIF END PREAMBLE EXTENSION ADDED BY LATEXDIFF

\begin{document}

\section{Introduction}

The vertical distribution of underwater light is an important driver of many aquatic processes such as primary production by phytoplankton and photochemical reactions like photodegradation of organic matter. Hence, an adequate description of the underwater light regime is mandatory to understand energy fluxes in aquatic ecosystems. In open water, when assuming an optically homogeneous water column, downward irradiance at any given wavelength follows quite well a monotonically exponential decrease with depth, which can be modeled as follows \citep{Kirk1994}:

\begin{equation}
    \edz{} = \edzero{} \times e^{-\ked(z)}
    \label{eq:edz}
\end{equation}

where \edz{} is the downward irradiance ($W~m^{-2}$) at depth $z$ (m), \edzero{} is the downward irradiance just below the surface and \ked{} is the diffuse vertical attenuation coefficient ($m^{-1}$) describing the rate at which light decreases with increasing depth. \ked{} is one of the most used apparent optical properties (AOP) of seawater and a precise estimation of this parameter is generally essential to measure or model primary production. For example, to determine primary production based on on-deck simulated incubations or photosynthetic parameters derived from photosynthesis vs. irradiance curves (P vs. E curves requires measured or estimated values of \ked{} (e.g. \citet{Morel1996}). Nowadays, \ked{} is relatively easy to estimate using commercially available radiometers. 

In the Arctic, a complex mosaic composed of ice, snow, leads, melt ponds and open water is characterizing the surface of ice-infested waters \citep{Nicolaus2013, Katlein2015, Katlein2016}. There, phytoplankton is exposed to a highly variable light regime while drifting under these features (e.g. \citet{Lange2017b}). Estimating primary production of phytoplankton under sea-ice requires an adequate approach that captures this large-area variability in the light field. In situ incubations at single locations of seawater samples inoculated with $^{14}$C or $^{13}$C are not appropriate because they reflect primary production under local light conditions, not representative or the range of irradiance experienced by drifting phytoplankton over a large area. One classical approach that is more adequate consists in conducting on-deck simulated 24h incubations of seawater samples inoculated with $^{14}$C or $^{13}$C and applying the average light attenuations at the depths of sample collection, using natural illumination and neutral filters. An alternative approach consists in calculating primary production using modeled or measured daily time series of incident irradiance, sea ice transmittance, and in-water vertical attenuation coefficients, combined with photosynthetic parameters determined on P vs. E curves measured with short ($\le$ 2h) incubations of seawater samples inoculated with $^{14}$C. Both approaches require that the vertical profile of the irradiance experienced by drifting phytoplankton be appropriately determined, which is challenging due to surface heterogeneity. Traditionally, one or very few \edz{} profiles are measured at discrete locations under sea ice \citep{Mundy2009}. Such parsimonious measurements, however, do not capture the variability induced by sea ice features. In recent studies, to better document the spatial variability of \edz{}, radiometers were attached to either remotely operated vehicles \citep{Katlein2015} or a SUIT, a net developed for deployment in ice covered waters, typically behind an icebreaker \citep{Lange2017b}. Both a ROV and the SUIT allow a better description of the light field under sea ice, which is more appropriate for determining average irradiance experienced by drifting phytoplankton. Such under-ice measurements can then be combined with \ked{} values to propagate light at depth. 

Propagating \edz{} using \ked{} values determined based on few discrete vertical profiles of \edz{} under sea-ice, a limitation that applies to any strategy for radiometer deployment, is however, very challenging because of surface heterogeneity. Indeed, under sea ice covered or not with snow,surrounded with for instance melt ponds, local \ed{} may not follow the usual monotonically exponential decrease with increasing depth (equation 1). Rather, irradiance just below sea ice few meters aside a melt pond increases with depth instead of decreasing and reaches a subsurface maximum between $\approx$5-20 meters depth \citep{Frey2011, Katlein2016, Laney2017}. Furthermore, two vertical light profiles measured few meters apart under sea ice are often very different. Hence, local measurements of light under heterogeneous sea ice do not allow an adequate description of the average light field as it would be seen by drifting phytoplankton cells at different depths. This makes estimations of primary production and the interpretation of biogeochemical data challenging in the presence of sea ice.

To fit vertical profiles of \edz{} that do not follow an exponential decay under sea ice covered with melt ponds,  \citet{Frey2011} proposed a simple geometric model (equation \ref{eq:frey2011}). 

\begin{equation}
    \edz{} = \pi \edzero{} (1 + P(N-1)\cos^2\phi)e^{-\ked(z)}
    \label{eq:frey2011}
\end{equation}

where \edzero{} is the irradiance directly below the ice/snow, $P$ the areal fraction of the ice cover, $N$ the ratio between ice and melt ponds transmittance and $\phi$ a fitting parameter defined as $\arctan(R/z)$ with $R$ the radius of the ice patch. An important drawback of this method is that additional field observations of $N$ and $P$ are required to adequately parameterize the model which makes its use more difficult. To address this concern, \citet{Laney2017} proposed a semi-empirical parameterization that includes a second exponential coefficient to equation \ref{eq:edz} to model light decrease between ice surface and ice-ocean interface.

\begin{equation}
    \edz{} = \edzero{} \times e^{-\ked(z)} - (\edzero{} - E_d(\text{NS})) \times e^{-K_{NS}(z)}
    \label{eq:laney2017}
\end{equation}

where \edzero{} is the irradiance that would be observed under homogeneous snow/ice cover, $E_d(\text{NS})$ is the irradiance under ice, $K_{NS}(z)$ describes the near-surface decrease of \edzero{}.  Both methods by \citet{Frey2011} and \citet{Laney2017} allow propagating local \edz{} vertically under specific sea ice features. Additionally, they in principle allow estimating KEd under homogeneous sea ice. What matters, when trying to determine primary production by phytoplankton that drift under sea ice and, therefore, is not static under some anecdotal sea ice feature, is the average shape of the vertical \edz{} profile, which may possibly be predictable using a large-area \meanked{} as under a wavy open-ocean surface \citep{Zaneveld2001}. 

In this study, using both in-situ data and 3D Monte-Carlo numerical simulations of radiative transfer, we show that the vertical propagation of average \edz{}, \meanedz{}, is reasonably well approximated by a single exponential decay with a so-called large-area \meanked{} under sea-ice covered with melt ponds. We further demonstrate that the large-area \meanked{} can be estimated from measurements of the vertical attenuation coefficient for upward radiance \klu{}, because the latter is supposedly less affected by local surface features of the ice cover. 
\section{Material and methods}

\subsection{Study site and field campaign}

The field campaign was part of the GreenEdge project (www.greenedgeproject.info)  which was conducted on landfast ice southeast of the Qikiqtarjuaq Island (67.4797N, -63.7895W). The field operations took place at an ice camp where the water depth was 360 m, from April 20 until July 27 of 2016 (Supplementary Fig. 1). During the sampling period, the study site experienced changes in the snow cover and lanfast ice thicknesses thickness between 0.32--49.00 and 105.75--149.31 cm, respectively.

\subsection{Underwater light measurements}

A total of 83 vertical light profiles using a factory calibrated  ICE-Pro (an ice floe version of the C-OPS - Compact-Optical Profiling System - from Biospherical Instruments Inc.) equipped with both downward irradiance \edz{} ($W~cm^{-2}$) and upward radiance \luz{} ($W~cm^{-2}~sr^{-1}$) radiometers were measured during the campaign. The IcePRO system is a negatively buoyant instrument with 10 inches in diameter cylindrical shape, and is not designed for free-fall casts (as opposed to its open water version). To perform the triplicate profiles, the frame is manually lowered in an auger hole that has been cleaned for ice chunks. Once underneath the ice layer, clean and fresh snow is shoveled back in the hole, to prevent any bright spot right on top of the sensors, and great care is taken not to pollute the hole surroundings (footsteps, water and slush spillage from the auger drilling, etc.). The operator then steps back 50 m, while keeping the sensors right under the ice, to avoid any human shadow on top of the profile. Then the frame is lowered manually at a constant descent rate of approximately 0.3 $m \times s^{-1}$. The above surface atmospheric reference sensor is fixed on a tripod standing on the floe (very steady), approximately 2 m above the surface and above any neighbour ice camp feature. Data processing and validation were performed using a protocol inspired by the one proposed by \citet{Smith1984} which is now used by various space agencies. Measurements were made at 19 wavelengths: 380, 395, 412, 443, 465, 490, 510, 532, 555, 560, 589, 625, 665, 683, 694, 710, 765, 780 and 875 nm. For this study, \ed{} and \lu{} spectra were interpolated linearly between 400 and 700 nm every 10 nm. In-situ attenuation coefficients ($K$) for both \ed{} (\ked{}) and \lu{} (\klu{}) were calculated on a 5 meters sliding window (10--15 m, 15--20 m, $\ldots$, 70--75 m, 75--80 m) starting at 10 meters to reduce the effects of surface heterogeneity. A total of 72 044 non-linear models were calculated to estimate $K$ from equation 1 (83 profiles $\times$ 14 depths $\times$ 31 wavelengths $\times$ 2 lights (\ed{}, \lu{})). A conservative $R^2$ of 0.99 was used to filter out poor models (i.e. noisy profiles that were not following an exponential decrease). 42 407 models were kept for subsequent analysis.

\subsection{3D Monte-Carlo numerical simulations}

\subsubsection{Theory and geometry}

3D numerical Monte-Carlo simulation is a convenient approach to model the light field under spatially heterogeneous sea \citep{Mobley_ocean_optics_book, Petrich2012, Katlein2014, Katlein2016}.  They are simple to understand, versatile, and incident light, inherent optical properties (IOPs) and geometry can be easily changed. In this study, we used SimulO, a Monte-Carlo software that allows simulating the propagation of light in various geometries from optical instruments to open or ice covered oceanic waters \citep{Leymarie2010}. Simulations were performed in an idealized ocean described by a cylinder of 120 m radius and 150 m depth. Given our interest in surface light profiles, the deepest software photons counter was placed at 25 m depth. The water IOPs were selected to reflect pre-bloom conditions (a = b = 0.05 $m^{-1}$) in the green/blue spectral region. These typical averaged values were measured in the visible range during the GreenEdge 2016 campaign using an in-situ spectrophotometer (ACS, Sea-Bird Scientific) and represent the contribution of both pure water and water’s constituents. The scattering phase function was described by a Fournier-Forand analytic form with a 3\% backscatter fraction \citep{Fournier1994, Mobley2002}. Sea ice was incorporated at the upper boundary of the ocean using a 2D emitting surface of 100 m radius. A 5 m radius melt pond was set-up at the center of the surface (Fig. 1). The photon emission of the melt pond surface has four times the intensity of the surrounding ice which corresponds to typical conditions found in Arctic during summer \citep{Perovich2016}. SimulO does not allow to use arbitrary  emission angular distribution. To overcome this problem, two lambertian sources of 90 and 60 degrees were summed up in order to mimic observed under ice radiance light field \citep{Girard2018} and reproduce the subsurface light maximums observed between $\approx$ 5--20 meters (supplementary Fig. 4). The same emission angular distribution was used for both ice and melt pond surfaces. For this purpose of this study, the small difference between the light field shape measured under melt pond vs ice is much less important compared the their difference of intensity \citep{Girard2018}.

2D horizontal detectors were placed vertically every 0.5 meters, up to 25 meters. Detectors include 1-$m^2$ pixels measuring planar irradiance for the downward face and radiance for the upward face (5 degrees half angle). In order to avoid the effect of the boundary (i.e. absorption by the side of the cylinder used to simulate the water column), data outside a radius of 50 meters were not used. A total number of 7.14e10 photons were simulated in order the obtained sufficient upwelling photons. Due to the low scattering coefficients used to reproduce in-situ conditions observed during the sampling campaign, radiance profiles were noisy because only a small number of upward photons could be captured. To address this issue, radiance profiles were smoothed out using Gaussian fittings (supplementary Fig. 5).  The simulation took approximately 6000 hours distributed over 2000 CPU cores. 

\subsubsection{Estimation of different reference light profiles}

Using the Monte-Carlo simulation, data were averaged accordingly to six different radius with therefore varying melt pond proportions to explore how melt pond influence the averaged underwater irradiance and radiance profiles (Fig. 1). This is equivalent to varying melt pond concentration. For each case, simulated light profiles were averaged within the following surface areas: (1) 10 meters radius (25\% melt pond cover), (2) 11.18 meters radius (20\% melt pond cover), (3) 12.91 meters radius (15\% melt pond cover), (4) 15.811 meters radius (10\% melt pond cover), (5) 22.361 meters radius (5\% melt pond cover) and (6) 50 meters radius (1\% melt pond cover). For each of these configurations, averaged light profile, \meanedz{}, was subsequently viewed as an adequate description of the average underwater light field. A total of 45 light profiles evenly spaced by one meter around the melt pond were further extracted to mimic local measurements of light and to calculate associated attenuation coefficients (colored circles in Fig. 1).

\subsection{Statistical analysis}

All statistical analysis and graphics were carried out with R 3.5.1 \citep{RCoreTeam2018}.
\section{Results}

\subsection{Comparing in situ \DIFaddbegin \DIFadd{downward }\DIFaddend irradiance (\ed{}) and \DIFaddbegin \DIFadd{upward }\DIFaddend radiance (\lu{}) measurements}

An example showing in situ downward irradiance (\ed{}) profiles and upward radiance (\lu{}) profiles at 16 \DIFdelbegin \DIFdel{different }\DIFdelend visible wavelengths measured under ice is presented in Fig. \DIFdelbegin \DIFdel{2. }\DIFdelend \DIFaddbegin \DIFadd{3. }\DIFaddend For the \ed{} profiles, subsurface light maxima at \DIFaddbegin \DIFadd{a depth of }\DIFaddend around 10 \DIFdelbegin \DIFdel{meters }\DIFdelend \DIFaddbegin \DIFadd{m }\DIFaddend are clearly visible between 400 and 560 nm. These peaks are not visible in the yellow/red \DIFdelbegin \DIFdel{regions }\DIFdelend \DIFaddbegin \DIFadd{region }\DIFaddend (580--700 nm). For the \lu{} profiles, no subsurface light maxima were found at any wavelength. To \DIFdelbegin \DIFdel{look closer }\DIFdelend \DIFaddbegin \DIFadd{have a closer look }\DIFaddend at the shape of both \ed{} and \lu{} light profiles, data below 10 m \DIFdelbegin \DIFdel{have been }\DIFdelend \DIFaddbegin \DIFadd{were }\DIFaddend normalized to the value at 10 m (Fig. \DIFdelbegin \DIFdel{3}\DIFdelend \DIFaddbegin \DIFadd{4}\DIFaddend ). Below 10 m and between 400 and 580 nm, both \ed{} and \lu{} profiles \DIFdelbegin \DIFdel{follow the same pattern }\DIFdelend \DIFaddbegin \DIFadd{presented the same shape (i.e. yield the same rate of extinction }\DIFaddend with increasing depth\DIFdelbegin \DIFdel{. At larger }\DIFdelend \DIFaddbegin \DIFadd{). At longer }\DIFaddend wavelengths ($\ge$ 600 nm), differences between the shapes of \ed{} and \lu{} profiles \DIFdelbegin \DIFdel{increase. A linear regression analysis between all in situ normalized }%DIFDELCMD < \ed{} %%%
\DIFdel{and }%DIFDELCMD < \lu{} %%%
\DIFdel{profiles shows that determination coefficients ($R^2$) range between 0.75 and 1 (Supplementary Fig. 3). A sharp decrease and a high variability of calculated $R^2$ occurred beyond 575 nm. This suggests a gradual decoupling between }%DIFDELCMD < \ed{} %%%
\DIFdel{and }%DIFDELCMD < \lu{} %%%
\DIFdel{profiles at larger wavelengths, likely due to the effect of inelastic scattering (mostly, Raman). }%DIFDELCMD < 

%DIFDELCMD < %%%
\DIFdelend \DIFaddbegin \DIFadd{increased. }\DIFaddend Irradiance and radiance \DIFaddbegin \DIFadd{diffuse }\DIFaddend attenuation coefficients (\ked{} and \DIFdelbegin %DIFDELCMD < \klu%%%
\DIFdelend \DIFaddbegin \klu{}\DIFaddend ) calculated on \DIFdelbegin \DIFdel{five meters depth layers }\DIFdelend \DIFaddbegin \DIFadd{layers of a 5 m depth }\DIFaddend are compared in Fig. \DIFdelbegin \DIFdel{4 }\DIFdelend \DIFaddbegin \DIFadd{5 }\DIFaddend for all 83 profiles. In the blue/green/yellow regions (400--580 nm), the determination coefficients between \DIFdelbegin %DIFDELCMD < \ked{} %%%
\DIFdel{and }%DIFDELCMD < \klu %%%
\DIFdelend \DIFaddbegin \klu{} \DIFadd{and }\ked{} \DIFaddend varied between 0.98 at the surface (10--15 m) and 0.64 at depth (\DIFdelbegin \DIFdel{75--80 }\DIFdelend \DIFaddbegin \DIFadd{75-80 }\DIFaddend m). For most of the surface \DIFdelbegin \DIFdel{layer}\DIFdelend \DIFaddbegin \DIFadd{layers}\DIFaddend , regression lines \DIFdelbegin \DIFdel{lined-up }\DIFdelend \DIFaddbegin \DIFadd{lined up }\DIFaddend with the 1:1 lines. Slight deviations from the 1:1 lines started to appear after 60 \DIFdelbegin \DIFdel{meters }\DIFdelend \DIFaddbegin \DIFadd{m }\DIFaddend where \ked{} was on average higher than \klu{}. \DIFdelbegin \DIFdel{Relationships }\DIFdelend \DIFaddbegin \DIFadd{The relationships }\DIFaddend including orange and red wavelengths are presented in Supplementary Fig. 3. \DIFaddbegin \DIFadd{A linear regression analysis between all in situ normalized }\ed{} \DIFadd{and }\lu{} \DIFadd{profiles showed that determination coefficients (}\rsquared{}\DIFadd{) range between 0.75 and 1 (Supplementary Fig. 4). A sharp decrease and a high variability of calculated }\rsquared{} \DIFadd{occurred beyond 575 nm. This suggests a gradual decoupling between }\ed{} \DIFadd{and }\lu{} \DIFadd{profiles at longer wavelengths, likely due to the effect of inelastic scattering (mostly, Raman). 
}\DIFaddend 

\DIFdelbegin \subsection{\DIFdel{3D Monte-Carlo numerical simulations}}
%DIFAUXCMD
\addtocounter{subsection}{-1}%DIFAUXCMD
\DIFdelend \DIFaddbegin \begin{figure}[H]
	\centering
	\includegraphics[scale = 0.8]{../../../../graphs/fig3.pdf}
	\caption{\DIFaddFL{Examples of in situ downward irradiance (}\edz{}\DIFaddFL{) and upward radiance (}\luz{}\DIFaddFL{) profiles taken under-ice on 2016-06-20. Note the subsurface maxima for the downward irradiance profiles that are not present in the upward radiance profiles.}}
\end{figure}
\DIFaddend 

\DIFdelbegin \subsubsection{\DIFdel{Simulated irradiance and radiance}}
%DIFAUXCMD
\addtocounter{subsubsection}{-1}%DIFAUXCMD
\DIFdelend \DIFaddbegin \begin{figure}[H]
	\centering
	\includegraphics[scale = 1]{../../../../graphs/fig4.pdf}
	\caption{\DIFaddFL{Comparison of downward irradiance (}\edz{}\DIFaddFL{) and upward radiance (}\luz{}\DIFaddFL{) for one light profile acquired under-ice. Profiles were normalized to the measured radiometric value at 10 m depth (under the subsurface light maximum) in order to emphasize the similar shape between }\edz{} \DIFaddFL{and }\luz{}\DIFaddFL{.}}
\end{figure}
\DIFaddend 

\DIFdelbegin \DIFdel{Under the melt pond, the relative density of irradiance photons was higher than that of the radiance photons (Fig. 5). Another key difference is that simulated }\DIFdelend \DIFaddbegin \begin{figure}[H]
	\centering
	\includegraphics[scale = 1]{../../../../graphs/fig5.pdf}
	\caption{\DIFaddFL{Scatter plots showing the relationships between }\ked{} \DIFaddFL{and }\klu{} \DIFaddFL{between 400 and 580 nm at different depths (numbers in gray boxes). Red lines represent the regression lines of the fitted linear models. Regression equations and determination coefficients ($R^2$) are also provided in each plot. Dashed lines are the 1:1 lines.}}
\end{figure}

\subsection{\DIFadd{3D Monte Carlo numerical simulations}}

\DIFadd{Fig. 6 shows cross-sections of the simulated downward irradiance and upward radiance. A key difference for the upcoming discussion is that the simulated upward }\DIFaddend radiance was more \DIFdelbegin \DIFdel{diffuse compared to irradiance. Over the simulated study area (50 meters radius, Fig. 1), a total of 403 irradianceprofiles with and without subsurface light maxima were extracted (Fig. 6). For these profiles, the number of "captured" photons for }%DIFDELCMD < \ed{} %%%
\DIFdel{varied between $4.8 \times 10^5$ and $12 \times 10^6$. Due to low scattering, the number of upward photons was much lower and ranged between 25 and 400 (Fig. 6). Averaged irradiance, (}%DIFDELCMD < \meanedz{}%%%
\DIFdel{), and radiance, (}%DIFDELCMD < \meanluz{}%%%
\DIFdel{), reference profilesshowed the same pattern where the number of photons increased with increasing melt pond coverage (Fig. 6)}\DIFdelend \DIFaddbegin \DIFadd{homogeneous compared to the simulated downward irradiance. Fig. 7 shows the reference irradiance, }\edz{}\DIFadd{, and reference radiance, }\luz{}\DIFadd{, profiles}\DIFaddend . The highest \DIFdelbegin \DIFdel{density of photons occured }\DIFdelend \DIFaddbegin \DIFadd{irradiance and radiance occurred }\DIFaddend when the melt pond occupied 25\% of the sampling area, allowing for more light to propagate in the water column. \DIFdelbegin \DIFdel{Note that none of the averaged }%DIFDELCMD < \meanedz{} %%%
\DIFdel{and }%DIFDELCMD < \meanluz{} %%%
\DIFdelend \DIFaddbegin \DIFadd{None of the }\edz{} \DIFadd{and }\luz{} \DIFaddend reference profiles showed subsurface light maxima\DIFdelbegin \DIFdel{(Fig. 6). }%DIFDELCMD < 

%DIFDELCMD < %%%
\subsubsection{\DIFdel{Attenuation coefficients and propagating light profiles}}
%DIFAUXCMD
\addtocounter{subsubsection}{-1}%DIFAUXCMD
%DIFDELCMD < 

%DIFDELCMD < %%%
\DIFdel{Some irradiance profiles extracted outside }\DIFdelend \DIFaddbegin \DIFadd{. Fig. 8 shows the 50 simulated local downward irradiance and upward radiance light profiles evenly spaced by 1 m from }\DIFaddend the melt pond \DIFdelbegin \DIFdel{at distances between 5 and 15 meters of the center (to mimic local radiometric measurements) showed the same subsurface light maxima (Fig. 7) as observed on in situ profiles (Fig. 2) . Beyond approximately 15 meters, }\DIFdelend \DIFaddbegin \DIFadd{centre. Local downward irradiance profiles under the melt pond (0--5 m) showed a rapid decrease with increasing depth described by a monotonically exponential or quasi-exponential decrease. Local simulated downward irradiance profiles just outside the melt pond (5--10 m) were characterized with }\DIFaddend subsurface light maxima \DIFdelbegin \DIFdel{disappeared and }\DIFdelend \DIFaddbegin \DIFadd{occurring at a depth of between approximately 5 and 10 m. Further away from the melt pond centre, downward }\DIFaddend irradiance profiles followed a monotonically exponential or quasi-exponential decrease\DIFdelbegin \DIFdel{(equation 1). Note that subsurface maxima were not found on radiance profiles as with in situ data. From both }\DIFdelend \DIFaddbegin \DIFadd{. None of the simulated upward radiance light profiles presented subsurface light maxima (Fig. 8). From local simulated }\DIFaddend irradiance and radiance profiles \DIFaddbegin \DIFadd{(Fig. 8)}\DIFaddend , \ked{} \DIFaddbegin \DIFadd{and }\klu{} \DIFadd{were calculated by fitting Equation 1 between 0 and 25 m. Results are presented in Fig. 9. }\ked{} \DIFaddend varied between 0.065 and \DIFdelbegin \DIFdel{0.096 $m^{-1}$ }\DIFdelend \DIFaddbegin \DIFadd{0.157 }\mminus{} \DIFaddend and \klu{} between 0.079 and \DIFdelbegin \DIFdel{0.1 $m^{-1}$ (Fig. 7, supplementary Fig. 6). }%DIFDELCMD < 

%DIFDELCMD < %%%
\DIFdel{Propagating surface reference light (}%DIFDELCMD < \edzero{}%%%
\DIFdel{, surface values of the colored }\DIFdelend \DIFaddbegin \DIFadd{0.116 }\mminus{}\DIFadd{. These }\ked{} \DIFadd{and }\klu{} \DIFadd{were used to propagate light downward from surface reference values }\edzero{}\DIFadd{. Fig. 10 shows the profiles resulting from this operation. A greater dispersion around the reference profiles (thick black }\DIFaddend lines in Fig. \DIFdelbegin \DIFdel{6) through the water column }\DIFdelend \DIFaddbegin \DIFadd{10) occurred when }\DIFaddend using \ked{} \DIFdelbegin \DIFdel{resulted in a greater variability }\DIFdelend compared to the profiles generated with \DIFaddbegin \DIFadd{similarly derived }\DIFaddend \klu{} \DIFdelbegin \DIFdel{(Fig. 8)}\DIFdelend \DIFaddbegin \DIFadd{values}\DIFaddend . The relative differences between \DIFdelbegin \DIFdel{both }\DIFdelend \DIFaddbegin \DIFadd{the }\DIFaddend depth-integrated \DIFdelbegin \DIFdel{(i.e. total number of photons) reference profiles and predicted profiles }\DIFdelend \DIFaddbegin \DIFadd{values of each local profiles (coloured lines in Fig. 10) and the depth-integrated values of the reference profiles (thick black lines in Fig. 10) }\DIFaddend were used to quantify the error of using either \ked{} or \DIFdelbegin %DIFDELCMD < \klu %%%
\DIFdelend \DIFaddbegin \klu{} \DIFaddend as a proxy to predict downward irradiance in the water column (Fig. \DIFdelbegin \DIFdel{9). Overall, the greatest errors on predictions reached approximately 38\% when using }\DIFdelend \DIFaddbegin \DIFadd{11). Below the melt pond, }\DIFaddend \ked{} \DIFdelbegin \DIFdel{at approximately }\DIFdelend \DIFaddbegin \DIFadd{overestimated the total downward irradiance by up to 40\% for the 1\% melt pond reference surface. In this region, the local $K$ coefficients are inflated. In the transition region, between }\DIFaddend 5 \DIFdelbegin \DIFdel{meters from ice ridge }\DIFdelend \DIFaddbegin \DIFadd{and 10 m from the centre of the melt pond, where subsurface maxima are observed, }\ked{} \DIFadd{underestimated the downward irradiance by up to 35\% for the 25\% melt pond reference surface. Further away from the edge }\DIFaddend of the melt pond\DIFaddbegin \DIFadd{, the errors saturated to maximum -25\%. The same behaviour is observed for }\klu{} \DIFadd{but with about two times less amplitude}\DIFaddend . The mean relative errors were lower \DIFaddbegin \DIFadd{by approximately a factor of two }\DIFaddend when using \klu{} (\DIFdelbegin \DIFdel{-8}\DIFdelend \DIFaddbegin \DIFadd{-7}\DIFaddend \%) compared to \ked{} (\DIFdelbegin \DIFdel{-17}\DIFdelend \DIFaddbegin \DIFadd{-12}\DIFaddend \%). \DIFdelbegin \DIFdel{The errors of the predictions }\DIFdelend \DIFaddbegin \DIFadd{Also, the prediction errors }\DIFaddend stabilized at a shorter distance from the \DIFdelbegin \DIFdel{melt pond ice ridge }\DIFdelend \DIFaddbegin \DIFadd{centre of the melt pond }\DIFaddend when using \klu{} ($\approx$ 10 \DIFdelbegin \DIFdel{meters}\DIFdelend \DIFaddbegin \DIFadd{m}\DIFaddend ) compared with using \ked{} ($\approx$20 \DIFdelbegin \DIFdel{meters). 
Furthermore, comparing the different spatial configurations (Fig. 1) showed that the largest error occurred when the melt pond occupied 25\% of the area used to derive the reference average profile (Fig. 9}\DIFdelend \DIFaddbegin \DIFadd{m). 
}

\begin{figure}[H]
	\centering
	\includegraphics[scale = 1]{../../../../graphs/fig6.pdf}
	\caption{\DIFaddFL{Cross-sections of simulated downward irradiance and upward radiance light fields under a melt pond with a 5 m radius. The logarithm of the normalized number of photons has been used to create the scale for visualization. The normalization has been done using the values modelled at 0.5 m depth and at a horizontal distance of 50 m from the center of the melt pond.}}
\end{figure}

\begin{figure}[H]
	\centering
	\includegraphics[scale = 1]{../../../../graphs/fig7.pdf}
	\caption{\DIFaddFL{Simulated reference downward irradiance and upward radiance profiles (}\meanedz{}\DIFaddFL{, }\meanluz{}\DIFaddFL{) for six different areas with varying proportions of the surface occupied by the melt pond (see Fig. 1). Note that none of the averaged irradiance profiles show the same subsurface light maxima as observed with in situ data (see Fig. 3).}}
\end{figure}


\begin{figure}[H]
	\centering
	\includegraphics[scale = 1]{../../../../graphs/fig8.pdf}
	\caption{\DIFaddFL{Simulated local downward irradiance and upward radiance profiles at different horizontal distances from the center of the melt pond (see Fig. 1) used to compute }\ked{} \DIFaddFL{and }\klu{}\DIFaddFL{. These attenuation coefficients were used to propagate surface reference downward irradiance (}\edzero{}\DIFaddFL{, the surface values of the lines in Fig. 7) through the water column.}}
\end{figure}

\begin{figure}[H]
	\centering
	\includegraphics[scale = 1]{../../../../graphs/fig9.pdf}
	\caption{\DIFaddFL{Diffuse attenuation coefficients calculated from local downward irradiance and upward radiance light profiles simulated at different distances from the center of the melt pond (see Fig. 8).}}
\end{figure}

\begin{figure}[H]
	\centering
	\includegraphics[scale = 0.8]{../../../../graphs/fig10.pdf}
	\caption{\DIFaddFL{Reference downward irradiance profiles (thick black lines) and propagated irradiance profiles (colored lines) using locals }\ked{} \DIFaddFL{and }\klu{} \DIFaddFL{(see Fig. 8). Light was propagated using the surface reference downward irradiance.}}
\end{figure}

\begin{figure}[H]
	\centering
	\includegraphics[scale = 1]{../../../../graphs/fig11.pdf}
	\caption{\DIFaddFL{Relative errors of the predictions calculated as the relative differences between the depth integral of the reference and predicted light profiles.}}
\end{figure}


\subsection{\DIFadd{Inelastic scattering}}

\DIFadd{Based on in situ data, our results have pointed out that }\klu{} \DIFadd{is not a good proxy for }\ked{} \DIFadd{at longer wavelengths (Supplementary Figs. 3-4) because of the effect of Raman scattering. To validate this hypothesis, we used the HydroLight (Sequoia Scientific, Inc.) radiative transfer numerical model to calculate theoretical downward irradiance and upward radiance and their associated vertical attenuation coefficients in an open water column in the presence of Raman scattering. The simulation was parameterized using IOPs measured during the field campaign (detailed information can be found in the supplementary section entitled Raman inelastic scattering). The simulation was able to reproduce the observed decoupling between }\ked{} \DIFadd{and }\klu{} \DIFadd{observed larger wavelengths $\ge$ 600 nm (Supplementary Fig. 5}\DIFaddend ).
\section{Discussion}

In the Arctic, melt pond coverage, lead coverage and ice/snow thickness can vary highly in both time and space \citep{Landy2014, Eicken2004}. Due to this sea ice heterogeneity, local under ice measurements of downward irradiance are often characterized by subsurface light maximums (Fig. 2). To model such profiles, \citet{Laney2017} proposed a semi-empirical parameterization using two exponential terms (see equation 3). Whereas their method might provide adequate estimations of instantaneous downward attenuation coefficients at specific locations, fitting a double exponential might not be ideal because data is modelled locally and do not provide an adequate description of the average light field (\meanedz{}) as it would be seen, for example, by drifting phytoplankton cells. In such conditions it was argued that under ice, irradiance measurements should be analyzed in the context of ice and surface properties within a radius of several meters since local measurements do not reproduce the full variability of the under ice light field \citep{Katlein2015}.

Using in-situ light measurements, it was found that \ed{} and \lu{} (and therefore \ked{} and \klu{}) were highly correlated bellow 10 meters (Fig. 3, Fig. 4), even when subsurface light maxima were present (Fig. 2). One possible explanation is that a \lu{} radiometer measures scattered light originating from a larger surface area, which reduce the effect of sea ice heterogeneity. Accordingly, no subsurface light maxima were observed in the in-situ radiance profiles. This reinforce the idea that \lu{} is less influenced by sea-ice surface heterogeneity. 

Based on Monte-Carlo simulations, our results showed that the average downward light profile, \meanedz, under heterogeneous sea ice cover follows a single term exponential function, even when melt ponds occupy a large fraction of the study area (Fig. 6). This is similar to what is observed under a wavy ice-free surface \citep{Zaneveld2001}. However, estimating \edz{} for a given area is not straightforward as it requires a large number of local profiles under the sea ice. An intuitive workaround to derive attenuation coefficient is to use upward radiance which is less influenced by sea surface heterogeneity compared to downward irradiance (Fig. 2, Fig. 3, Fig. 4). Monte-Carlo simulations showed that a local estimation of \klu{} could be a good proxy for \meanked{}. Accordingly, our results showed that propagating under sea ice average irradiance (\edzero{}) using \klu{} rather than \ked{} provided better estimations of the average downward profile (Fig. 8, Fig. 9).

There are at least two main factors influencing the quality of in-situ downward measurements under heterogeneous sea ice. The first factor is the horizontal distance from the melt pond ridge. Although the relative error of propagating \edzero{} using both \ked{} and \klu{} showed the same pattern, the largest error occurred when using local estimations of \ked{} made between 1 and 10 meters outside the melt pond (Fig. 9). In contrast, in the vicinity of the melt pond, the relative errors associated to the use of \klu{} was much lower and stabilized just after approximately 5 meters. The second factor driving the relative error of local measurements is the proportion occupied by melt ponds over the area of interest (Fig. 9). Indeed, higher proportions of melt pond allows for more light to penetrate in the water column. Hence, local measurements made under surrounding ice are more likely to show subsurface light maxima (see \citet{Frey2011}). Accordingly, when melt ponds accounted for 1\% of the total area, averaged errors in \edz{} using \klu{} was 1.33\% but increased to 18\% when the melt pond occupied 25\% of the total area (Fig. 9).
\section{Conclusions}

Our results show that under spatially heterogeneous sea ice surface, the average light profile, \meanedz{}, is well reproduced by a single exponential function. We also showed that propagating \edzero{} using \klu{} is better a choice compared to \ked{} under heterogeneous sea ice. Nowadays, radiance measurements are becoming more routinely measured during field campaigns of ecological studies, so we argue that one should use \klu{} when available to propagate \edzero{}. The main difficulty remains at finding good estimates of \edzero{}. In the last years, this became easier with the development of remotely operated vehicles \citep{Katlein2015, Arndt2017, Nicolaus2013}, remote sensing techniques and drone imagery. In this study we used a Monte-Carlo approach to model an idealized surface with a single melt pond (Fig. 4). However, in the Arctic, a complex mosaic composed of ice, snow, leads, melt ponds and open water is characterizing the landscape. Hence, further work using 3D radiative transfer models are needed to fully understand light distribution under spatially heterogeneous surface.

\clearpage
%%%%%%%%%%%%%%%%%%%%%%%%%%%%%%%%%%%%%%%%%%
\authorcontributions{Conceptualization, Philippe Massicotte (PM), Guislain Bécu (GB), Simon-Girard Lambert (SGL), Edouard Leymarie (EL) and Marcel Babin (MB); methodology, PM, GB, SGL, EL and MB; field work, GB, SGL, and MB; writing—original draft preparation, PM; writing—review and editing, PM, GB, SGL, EL and MB; supervision, MB; funding acquisition, MB}

%%%%%%%%%%%%%%%%%%%%%%%%%%%%%%%%%%%%%%%%%%
\DIFdelbegin %DIFDELCMD < \acknowledgments{The GreenEdge project is funded by the following French and Canadian programs and agencies: ANR (contract \#111112), CNES (project \#131425), IPEV (project \#1164), CSA, Fondation Total, ArcticNet, LEFE and the French Arctic Initiative (GreenEdge project). This project would not have been possible without the support of the Hamlet of Qikiqtarjuaq and the members of the community as well as the Inuksuit School and its principal, Jacqueline Arsenault. The project is conducted under the scientific coordination of the Canada Excellence Research Chair on Remote Sensing of Canada’s New Arctic Frontier and the CNRS and Université Laval Takuvik Joint International laboratory (UMI3376). The field campaign was successful thanks to the contributions of J. Ferland, G. Bécu, C. Marec, J. Lagunas-Morales, F. Bruyant, J. Larivière, E. Rehm, S. Lambert-Girard, C. Aubry, C. Lalande, A. LeBaron, C. Marty, J. Sansoulet, D. Christiansen-Stowe, A. Wells, M. Benoît-Gagné, E. Devred and M.-H. Forget from the Takuvik laboratory, C.J. Mundy and V. Galindo from University of Manitoba, and F. Pinczon du Sel and E. Brossier from Vagabond. We also thank Michel Gosselin, Québec-Océan, the CCGS Amundsen and the Polar Continental Shelf Program for their in-kind contribution in polar logistics and scientific equipment. This research was enabled in part by support provided by Calcul Québec (www.calculquebec.ca) and Compute Canada (www.computecanada.ca). S. L. Girard was supported by a postdoctoral fellowship from the Natural Sciences and Engineering Research Council of Canada (NSERC). We also acknowledge the Canada First Research Excellence Fund and the Sentinel North Strategy for their financial support.}
%DIFDELCMD < %%%
\DIFdelend \DIFaddbegin \acknowledgments{The GreenEdge project is funded by the following French and Canadian programs and agencies: ANR (contract \#111112), CNES (project \#131425), IPEV (project \#1164), CSA, Fondation Total, ArcticNet, LEFE and the French Arctic Initiative (GreenEdge project). This project would not have been possible without the support of the Hamlet of Qikiqtarjuaq and the members of the community as well as the Inuksuit School and its principal, Jacqueline Arsenault. The project is conducted under the scientific coordination of the Canada Excellence Research Chair on Remote Sensing of Canada’s New Arctic Frontier and the CNRS and Université Laval Takuvik Joint International laboratory (UMI3376). The field campaign was successful thanks to the contributions of J. Ferland, G. Bécu, C. Marec, J. Lagunas-Morales, F. Bruyant, J. Larivière, E. Rehm, S. Lambert-Girard, C. Aubry, C. Lalande, A. LeBaron, C. Marty, J. Sansoulet, D. Christiansen-Stowe, A. Wells, M. Benoît-Gagné, E. Devred and M.-H. Forget from the Takuvik laboratory, C.J. Mundy and V. Galindo from University of Manitoba, and F. Pinczon du Sel and E. Brossier from Vagabond. We also thank Michel Gosselin, Québec-Océan, the CCGS Amundsen and the Polar Continental Shelf Program for their in-kind contribution in polar logistics and scientific equipment. This research was enabled in part by support provided by Calcul Québec (www.calculquebec.ca) and Compute Canada (www.computecanada.ca). S. L. Girard was supported by a postdoctoral fellowship from the Natural Sciences and Engineering Research Council of Canada (NSERC). We also acknowledge the Canada First Research Excellence Fund and the Sentinel North Strategy for their financial support. We thank Dr. Dariusz Stramski and one anonymous reviewer for their valuable comments which helped to greatly improve the manuscript.}
\DIFaddend 

%%%%%%%%%%%%%%%%%%%%%%%%%%%%%%%%%%%%%%%%%%
\conflictsofinterest{The authors declare no conflict of interest.}

\section*{References}
\externalbibliography{yes}
\bibliography{/home/pmassicotte/Documents/library}
% \bibliography{C:/Users/pmass/Documents/library}


%%%%%%%%%%%%%%%%%%%%%%%%%%%%%%%%%%%%%%%%%%
\DIFaddbegin 

\section*{}
\includepdf[pages=-]{../../figures/figures.pdf}

\section*{}
\includepdf[pages=-]{../../appendix/appendix.pdf}


 \DIFaddend\end{document}

