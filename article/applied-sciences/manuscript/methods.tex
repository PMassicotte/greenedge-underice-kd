\section{Material and methods}

\subsection{Study site and field campaign}

The field campaign was part of the GreenEdge project (www.greenedgeproject.info)  which was conducted on landfast ice southeast of the Qikiqtarjuaq Island (67.4797N, -63.7895W). The field operations took place at an ice camp where the water depth was 360 m, from April 20 until July 27 of 2016 (Supplementary Fig. 1). During the sampling period, the study site experienced changes in the snow cover and lanfast ice thicknesses thickness between 0.32--49.00 and 105.75--149.31 cm, respectively.

\subsection{Underwater light measurements}

A total of 83 vertical light profiles using a factory calibrated  ICE-Pro (an ice floe version of the C-OPS - Compact-Optical Profiling System - from Biospherical Instruments Inc.) equipped with both downward irradiance \edz{} ($W~cm^{-2}$) and upward radiance \luz{} ($W~cm^{-2}~sr^{-1}$) radiometers were measured during the campaign. The IcePRO system is a negatively buoyant instrument with 10 inches in diameter cylindrical shape, and is not designed for free-fall casts (as opposed to its open water version). To perform the triplicate profiles, the frame is manually lowered in an auger hole that has been cleaned for ice chunks. Once underneath the ice layer, clean and fresh snow is shoveled back in the hole, to prevent any bright spot right on top of the sensors, and great care is taken not to pollute the hole surroundings (footsteps, water and slush spillage from the auger drilling, etc.). The operator then steps back 50 m, while keeping the sensors right under the ice, to avoid any human shadow on top of the profile. Then the frame is lowered manually at a constant descent rate of approximately 0.3 $m \times s^{-1}$. The above surface atmospheric reference sensor is fixed on a tripod standing on the floe (very steady), approximately 2 m above the surface and above any neighbour ice camp feature. Data processing and validation were performed using a protocol inspired by the one proposed by \citet{Smith1984} which is now used by various space agencies. Measurements were made at 19 wavelengths: 380, 395, 412, 443, 465, 490, 510, 532, 555, 560, 589, 625, 665, 683, 694, 710, 765, 780 and 875 nm. For this study, \ed{} and \lu{} spectra were interpolated linearly between 400 and 700 nm every 10 nm. In-situ attenuation coefficients ($K$) for both \ed{} (\ked{}) and \lu{} (\klu{}) were calculated on a 5 meters sliding window (10--15 m, 15--20 m, $\ldots$, 70--75 m, 75--80 m) starting at 10 meters to reduce the effects of surface heterogeneity. A total of 72 044 non-linear models were calculated to estimate $K$ from equation 1 (83 profiles $\times$ 14 depths $\times$ 31 wavelengths $\times$ 2 lights (\ed{}, \lu{})). A conservative $R^2$ of 0.99 was used to filter out poor models (i.e. noisy profiles that were not following an exponential decrease). 42 407 models were kept for subsequent analysis.

\subsection{3D Monte-Carlo numerical simulations}

\subsubsection{Theory and geometry}

3D numerical Monte-Carlo simulation is a convenient approach to model the light field under spatially heterogeneous sea \citep{Mobley_ocean_optics_book, Petrich2012, Katlein2014, Katlein2016}.  They are simple to understand, versatile, and incident light, inherent optical properties (IOPs) and geometry can be easily changed. In this study, we used SimulO, a Monte-Carlo software that allows simulating the propagation of light in various geometries from optical instruments to open or ice covered oceanic waters \citep{Leymarie2010}. Simulations were performed in an idealized ocean described by a cylinder of 120 m radius and 150 m depth. Given our interest in surface light profiles, the deepest software photons counter was placed at 25 m depth. The water IOPs were selected to reflect pre-bloom conditions (a = b = 0.05 $m^{-1}$) in the green/blue spectral region. These typical averaged values were measured in the visible range during the GreenEdge 2016 campaign using an in-situ spectrophotometer (ACS, Sea-Bird Scientific) and represent the contribution of both pure water and water’s constituents. The scattering phase function was described by a Fournier-Forand analytic form with a 3\% backscatter fraction \citep{Fournier1994, Mobley2002}. Sea ice was incorporated at the upper boundary of the ocean using a 2D emitting surface of 100 m radius. A 5 m radius melt pond was set-up at the center of the surface (Fig. 1). The photon emission of the melt pond surface has four times the intensity of the surrounding ice which corresponds to typical conditions found in Arctic during summer \citep{Perovich2016}. SimulO does not allow to use arbitrary  emission angular distribution. To overcome this problem, two lambertian sources of 90 and 60 degrees were summed up in order to mimic observed under ice radiance light field \citep{Girard2018} and reproduce the subsurface light maximums observed between $\approx$ 5--20 meters (supplementary Fig. 4). The same emission angular distribution was used for both ice and melt pond surfaces. For this purpose of this study, the small difference between the light field shape measured under melt pond vs ice is much less important compared the their difference of intensity \citep{Girard2018}.

2D horizontal detectors were placed vertically every 0.5 meters, up to 25 meters. Detectors include 1-$m^2$ pixels measuring planar irradiance for the downward face and radiance for the upward face (5 degrees half angle). In order to avoid the effect of the boundary (i.e. absorption by the side of the cylinder used to simulate the water column), data outside a radius of 50 meters were not used. A total number of 7.14e10 photons were simulated in order the obtained sufficient upwelling photons. Due to the low scattering coefficients used to reproduce in-situ conditions observed during the sampling campaign, radiance profiles were noisy because only a small number of upward photons could be captured. To address this issue, radiance profiles were smoothed out using Gaussian fittings (supplementary Fig. 5).  The simulation took approximately 6000 hours distributed over 2000 CPU cores. 

\subsubsection{Estimation of different reference light profiles}

Using the Monte-Carlo simulation, data were averaged accordingly to six different radius with therefore varying melt pond proportions to explore how melt pond influence the averaged underwater irradiance and radiance profiles (Fig. 1). This is equivalent to varying melt pond concentration. For each case, simulated light profiles were averaged within the following surface areas: (1) 10 meters radius (25\% melt pond cover), (2) 11.18 meters radius (20\% melt pond cover), (3) 12.91 meters radius (15\% melt pond cover), (4) 15.811 meters radius (10\% melt pond cover), (5) 22.361 meters radius (5\% melt pond cover) and (6) 50 meters radius (1\% melt pond cover). For each of these configurations, averaged light profile, \meanedz{}, was subsequently viewed as an adequate description of the average underwater light field. A total of 45 light profiles evenly spaced by one meter around the melt pond were further extracted to mimic local measurements of light and to calculate associated attenuation coefficients (colored circles in Fig. 1).

\subsection{Statistical analysis}

All statistical analysis and graphics were carried out with R 3.5.1 \citep{RCoreTeam2018}.