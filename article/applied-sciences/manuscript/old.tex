\documentclass[10pt,a4paper,leqno]{article}
\usepackage[T1]{fontenc}
\usepackage[utf8]{inputenc}
\usepackage{textcomp}
\usepackage{lmodern}
\usepackage[english]{babel}
\usepackage[autostyle]{csquotes}

\usepackage{mathptmx}

\usepackage{authblk}
\title{Estimating underwater light regime under spatially heterogeneous sea ice in the Arctic}
\author[1]{Philippe Massicotte}
\author[1]{Guislain Bécu}
\author[1]{Simon-Lambert Gigard}
\author[2]{Edouard Leymarie}
\author[1]{Marcel Babin}

\affil[1]{Takuvik Joint International Laboratory (UMI 3376) \protect\\ Université Laval (Canada) \& Centre National de la Recherche Scientifique (France)}
\affil[2]{Laboratoire d'Océanographie de Villefranche - LOV (UMR-CNRS 7093)}

%************************************************
% Bibliography
%************************************************

\usepackage{csquotes}

\PassOptionsToPackage{
	natbib=true,
	sorting=ynt,
	style=authoryear-comp,
	hyperref=true,
	backend=biber,
	maxbibnames=999,
	firstinits=true,
	uniquename=false,
	parentracker=true,
	url=false,
	doi=false,
	isbn=false,
	eprint=false,
	backref=false,
	sortcites,
}   {biblatex}
\usepackage{biblatex}

\DeclareLanguageMapping{english}{english-apa}
\addbibresource{/home/pmassicotte/Documents/library.bib}

\AtEveryBibitem{\clearfield{issn}}
\AtEveryCitekey{\clearfield{issn}}
\AtEveryBibitem{\clearfield{url}}
\AtEveryCitekey{\clearfield{url}}
\AtEveryBibitem{\clearfield{doi}}
\AtEveryCitekey{\clearfield{doi}}

\usepackage{lineno}
\renewcommand\linenumberfont{\normalfont\bfseries}

%\newcommand{\ked}{sdf}
\newcommand{\ked}{\ensuremath{K_{Ed}}}
\newcommand{\klu}{\ensuremath{K_{Lu}}}
\newcommand{\edz}{\ensuremath{{Ed(z)}}}
\newcommand{\meanedz}{\ensuremath{{\overline{Ed}(z)}}}

% Spacing
\setlength{\parindent}{4em}
\setlength{\parskip}{1em}
\renewcommand{\baselinestretch}{1.5}


\date{}
\begin{document}
\maketitle

\begin{abstract}
	The vertical attenuation coefficient for downward irradiance (\ked) is an apparent optical property commonly used in primary production models to propagation incident solar radiation in the water column. Measuring \ked is relatively trivial in open waters. In ice-covered waters, however, the spatially heterogeneous incident light field resulting from sea ice and related features (ridges, snow, melt ponds, leads) makes its determination challenging because downward irradiance light profiles are characterized by subsurface light maximums between $\approx$5-20 meters depth.  Using both in-situ data and 3D Monte-Carlo numerical simulations, we show that: (1) the average downward irradiance profile (\meanedz) underwater light profile under heterogeneous sea ice cover can be represented by a single-term exponential function and: (2) the vertical attenuation coefficient for upward radiance (\klu), which is up to two times less influenced by an heterogeneous incident light field than \ked in the vicinity of a melt pond, can be used to parameterize the \edz profile.
\end{abstract}

\clearpage

\section*{Introduction}
\linenumbers

The vertical distribution of underwater light is an important driver of many aquatic processes such as primary production by phytoplankton and photochemical reactions like photodegradation of organic matter. Hence, an adequate description of the underwater light regime is mandatory to understand energy fluxes in aquatic ecosystems. In open water, assuming an optically homogeneous water column, downward irradiance follows a monotonically exponential decrease with depth which can be modeled as follows \citep{Kirk1994}:

\clearpage
\printbibliography
\end{document}