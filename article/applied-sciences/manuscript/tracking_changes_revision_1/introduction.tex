\section{Introduction}

The vertical distribution of underwater light is an important driver of many aquatic processes\DIFaddbegin \DIFadd{, }\DIFaddend such as primary production by phytoplankton\DIFaddbegin \DIFadd{, }\DIFaddend and photochemical reactions\DIFdelbegin \DIFdel{like }\DIFdelend \DIFaddbegin \DIFadd{, such as the }\DIFaddend photodegradation of organic matter. Hence, an adequate description of the underwater light regime is mandatory to understand energy fluxes in aquatic ecosystems. In open water, when assuming an optically homogeneous water column, downward irradiance at any given wavelength follows\DIFaddbegin \DIFadd{, as a first approximation, }\DIFaddend quite well a monotonically exponential decrease with depth, which can be \DIFdelbegin \DIFdel{modeled }\DIFdelend \DIFaddbegin \DIFadd{modelled }\DIFaddend as follows \citep{Kirk1994} \DIFaddbegin \DIFadd{(Equation \ref{eq:edz})}\DIFaddend :

\begin{equation}
    \edz{} = \edzero{}\DIFdelbegin \DIFdel{\times }\DIFdelend \DIFaddbegin \DIFadd{~}\DIFaddend e\DIFdelbegin \DIFdel{^{-\ked(z)}
    }\DIFdelend \DIFaddbegin \DIFadd{^{-\ked(z)~z}
    }\DIFaddend \label{eq:edz}
\end{equation}

\DIFaddbegin \noindent \DIFaddend where \edz{} is the downward irradiance (\DIFdelbegin \DIFdel{$W~m^{-2}$}\DIFdelend \DIFaddbegin \wmsquare{}\DIFaddend ) at depth $z$ (m), \edzero{} is the downward irradiance \DIFaddbegin \DIFadd{(}\wmsquare{}\DIFadd{) }\DIFaddend just below the surface and \DIFdelbegin %DIFDELCMD < \ked{} %%%
\DIFdelend \DIFaddbegin \DIFadd{$K_d(z)$ }\DIFaddend is the diffuse vertical attenuation coefficient (\DIFdelbegin \DIFdel{$m^{-1}$}\DIFdelend \DIFaddbegin \mminus{}\DIFaddend ) describing the rate at which light decreases with increasing depth. \ked{} is one of the most \DIFaddbegin \DIFadd{commonly }\DIFaddend used apparent optical properties (AOP) of seawater\DIFdelbegin \DIFdel{and a precise }\DIFdelend \DIFaddbegin \DIFadd{, and a good }\DIFaddend estimation of this parameter is \DIFdelbegin \DIFdel{generally essential to measure or model }\DIFdelend \DIFaddbegin \DIFadd{important for measuring or modelling }\DIFaddend primary production. \DIFaddbegin \ked{} \DIFadd{may vary with depth because changes in inherent optical properties and/or in the structure of the light field. But as \mbox{%DIFAUXCMD
\citet{Kirk1994} }\hspace{0pt}%DIFAUXCMD
pointed out, for practical considerations in oceanography and limnology, the }\ked{} \DIFadd{value, even averaged in the euphotic zone, is a useful and valuable way to represent the downward irradiance attenuation in that upper layer. }\DIFaddend For example, to determine primary production based on \DIFdelbegin \DIFdel{on-deck simulated }\DIFdelend \DIFaddbegin \DIFadd{simulated on-deck }\DIFaddend incubations or photosynthetic parameters derived from photosynthesis vs. irradiance curves (P\DIFdelbegin \DIFdel{vs. E curves }\DIFdelend \DIFaddbegin \DIFadd{~vs.~E~curves) }\DIFaddend requires measured or estimated values of \ked{} (e.g. \citet{Morel1996}). Nowadays, \ked{} is relatively easy to estimate using commercially available radiometers.

\DIFdelbegin \DIFdel{In the Arctic , }\DIFdelend \DIFaddbegin \DIFadd{The ice-infested regions of the Arctic ocean are characterized by }\DIFaddend a complex mosaic \DIFdelbegin \DIFdel{composed of ice, snow, leads, melt pondsand open water is characterizing the surface of ice-infested waters }\DIFdelend \DIFaddbegin \DIFadd{made of sea ice with snow, melt ponds, ridges and leads }\DIFaddend \citep{Nicolaus2013, Katlein2015, Katlein2016}. \DIFdelbegin \DIFdel{There, phytoplankton }\DIFdelend \DIFaddbegin \DIFadd{Phytoplankton }\DIFaddend is exposed to a highly variable light regime while drifting under these features (e.g. \DIFdelbegin \DIFdel{\mbox{%DIFAUXCMD
\citet{Lange2017b}}\hspace{0pt}%DIFAUXCMD
}\DIFdelend \DIFaddbegin \DIFadd{\mbox{%DIFAUXCMD
\citet{Lange2017}}\hspace{0pt}%DIFAUXCMD
}\DIFaddend ). Estimating primary production of phytoplankton under \DIFdelbegin \DIFdel{sea-ice requires an adequate approach that captures }\DIFdelend \DIFaddbegin \DIFadd{sea ice requires an approach that is adequate to capture }\DIFaddend this large-area variability in the light field. In situ incubations at single locations of seawater samples inoculated with $^{14}$C or $^{13}$C are not appropriate because they reflect primary production under local light conditions, \DIFdelbegin \DIFdel{not representative or }\DIFdelend \DIFaddbegin \DIFadd{which is not representative of }\DIFaddend the range of irradiance experienced by drifting phytoplankton over a large area. One classical approach that is more adequate consists in conducting on-deck simulated \DIFdelbegin \DIFdel{24h }\DIFdelend \DIFaddbegin \DIFadd{24-hours }\DIFaddend incubations of seawater samples inoculated with $^{14}$C or $^{13}$C and applying the \DIFdelbegin \DIFdel{average light attenuations }\DIFdelend \DIFaddbegin \DIFadd{light attenuation }\DIFaddend at the depths of sample \DIFdelbegin \DIFdel{collection}\DIFdelend \DIFaddbegin \DIFadd{collections}\DIFaddend , using natural illumination and neutral filters. An alternative approach consists in calculating primary production using \DIFdelbegin \DIFdel{modeled }\DIFdelend \DIFaddbegin \DIFadd{modelled }\DIFaddend or measured daily time series of incident irradiance, sea ice transmittance \DIFdelbegin \DIFdel{, }\DIFdelend and in-water vertical attenuation coefficients, combined with photosynthetic parameters determined on P vs. E curves measured with short (\DIFdelbegin \DIFdel{$\le$ 2h}\DIFdelend \DIFaddbegin \DIFadd{under two hours}\DIFaddend ) incubations of seawater samples inoculated with $^{14}$C. \DIFdelbegin \DIFdel{Both approaches }\DIFdelend \DIFaddbegin \DIFadd{The latter two methods }\DIFaddend require that the vertical profile of the irradiance experienced by drifting phytoplankton be appropriately determined, which is challenging due to surface heterogeneity. Traditionally, one or very few \edz{} profiles are measured at discrete locations under sea ice \DIFdelbegin \DIFdel{\mbox{%DIFAUXCMD
\citep{Mundy2009}}\hspace{0pt}%DIFAUXCMD
. Such parsimonious }\DIFdelend \DIFaddbegin \DIFadd{(e.g. \mbox{%DIFAUXCMD
\citet{Mundy2009}}\hspace{0pt}%DIFAUXCMD
). Such }\DIFaddend measurements, however, do not capture the variability induced by sea ice features. In recent studies, to better document the spatial variability of \edz{}, radiometers were attached to either remotely operated vehicles \DIFaddbegin \DIFadd{(ROV) }\DIFaddend \citep{Katlein2015} or a \DIFdelbegin \DIFdel{SUIT}\DIFdelend \DIFaddbegin \DIFadd{surface and under-ice trawl (SUIT)}\DIFaddend , a net developed for deployment in \DIFdelbegin \DIFdel{ice covered }\DIFdelend \DIFaddbegin \DIFadd{ice-covered }\DIFaddend waters, typically behind an icebreaker \DIFdelbegin \DIFdel{\mbox{%DIFAUXCMD
\citep{Lange2017b}}\hspace{0pt}%DIFAUXCMD
. Both a ROV and the }\DIFdelend \DIFaddbegin \DIFadd{\mbox{%DIFAUXCMD
\citep{Lange2017}}\hspace{0pt}%DIFAUXCMD
. Both an ROV and a }\DIFaddend SUIT allow a better description of the light field \DIFaddbegin \DIFadd{right }\DIFaddend under sea ice, which is more appropriate for determining average irradiance experienced by drifting phytoplankton. Such under-ice measurements can then be combined with \DIFaddbegin \DIFadd{averaged }\DIFaddend \ked{} values to propagate light at depth.

\DIFdelbegin \DIFdel{Propagating }%DIFDELCMD < \edz{} %%%
\DIFdelend \DIFaddbegin \DIFadd{Estimating irradiance at depth for primary production measurement or calculation }\DIFaddend using \ked{} values \DIFdelbegin \DIFdel{determined based on }\DIFdelend \DIFaddbegin \DIFadd{derived from only a }\DIFaddend few discrete vertical profiles of \edz{} under \DIFdelbegin \DIFdel{sea-ice, a limitation that applies to any strategy }\DIFdelend \DIFaddbegin \DIFadd{heterogenous sea ice is problematic whatever the platform }\DIFaddend for radiometer deployment\DIFaddbegin \DIFadd{. Let us consider that phytoplankton, by continuously drifting horizontally relative to sea ice, is exposed to fluctuations in irradiance due to surface heterogeneity}\DIFaddend , \DIFdelbegin \DIFdel{is however, very challenging because of surface heterogeneity. Indeed, }\DIFdelend \DIFaddbegin \DIFadd{and that the relevant light metrics for primary production in such conditions is irradiance at any depth averaged over some horizontal area. When measuring an irradiance profile at one given location }\DIFaddend under sea ice\DIFdelbegin \DIFdel{covered or not with snow, surrounded with for instance melt ponds, local }%DIFDELCMD < \ed{} %%%
\DIFdelend \DIFaddbegin \DIFadd{, as the depth of the upward-looking detector increases, light from a larger area on the underside of the ice enters the detector field of view. In other words, the detector "sees" different things at different depths. One consequence is that }\edz{} \DIFadd{measured that way }\DIFaddend may not follow the usual monotonically exponential decrease with increasing depth (\DIFdelbegin \DIFdel{equation }\DIFdelend \DIFaddbegin \DIFadd{Equation }\DIFaddend 1). \DIFdelbegin \DIFdel{Rather, irradiance just below sea ice few meters aside }\DIFdelend \DIFaddbegin \DIFadd{For example, irradiance profiles measured beneath low-transmission sea ice (e.g. white ice) relative to surrounding areas showing melt ponds, show subsurface light maxima. The literature reports subsurface maxima varying between five and 15 m in depth \mbox{%DIFAUXCMD
\citep{Frey2011, Katlein2016, Laney2017}}\hspace{0pt}%DIFAUXCMD
. Conversely, it is also important to note that }\ked{} \DIFadd{estimations are biased when profiles are measured beneath an area of high transmission (e.g. }\DIFaddend a melt pond\DIFdelbegin \DIFdel{increases with depthinstead of decreasing and reaches a subsurface maximum between $\approx$5-20 meters depth \mbox{%DIFAUXCMD
\citep{Frey2011, Katlein2016, Laney2017}}\hspace{0pt}%DIFAUXCMD
. Furthermore}\DIFdelend \DIFaddbegin \DIFadd{) relative to surrounding areas \mbox{%DIFAUXCMD
\citep{Katlein2016}}\hspace{0pt}%DIFAUXCMD
. Indeed, with depth, light decreases more quickly than what would be expected from the inherent optical properties (IOPs) of the water column. In the field, this situation is more difficult to identify compared to profiles showing subsurface maxima because the former measurements may appear to follow a single exponential decrease but would not produce a diffuse attenuation coefficient that adequately describes the water mass. So}\DIFaddend , two vertical light profiles measured \DIFdelbegin \DIFdel{few meters }\DIFdelend \DIFaddbegin \DIFadd{a few metres }\DIFaddend apart under sea ice are often very different. \DIFdelbegin \DIFdel{Hence}\DIFdelend \DIFaddbegin \DIFadd{More importantly}\DIFaddend , local measurements of light under heterogeneous sea ice do not \DIFdelbegin \DIFdel{allow }\DIFdelend \DIFaddbegin \DIFadd{provide }\DIFaddend an adequate description of the average light field as it would be seen by drifting phytoplankton cells at different depths. This makes estimations of primary production and the interpretation of biogeochemical data challenging in the presence of sea ice.

To fit vertical profiles of \edz{} \DIFaddbegin \DIFadd{under bare ice }\DIFaddend that do not follow an exponential decay under sea ice covered with melt ponds, \citet{Frey2011} \DIFdelbegin \DIFdel{proposed }\DIFdelend \DIFaddbegin \DIFadd{proposes }\DIFaddend a simple geometric model (\DIFdelbegin \DIFdel{equation }\DIFdelend \DIFaddbegin \DIFadd{Equation }\DIFaddend \ref{eq:frey2011}). 

\begin{equation}
    \edz{} = \pi \edzero{} (1 + P(N-1)\cos^2\phi)e\DIFdelbegin \DIFdel{^{-\ked(z)}
    }\DIFdelend \DIFaddbegin \DIFadd{^{-\ked(z)~z}
    }\DIFaddend \label{eq:frey2011}
\end{equation}

\DIFaddbegin \noindent \DIFaddend where \edzero{} is the irradiance directly below the ice/snow, $P$ the areal fraction of the ice cover, $N$ the ratio between ice and melt ponds transmittance and $\phi$ a fitting parameter defined as $\arctan(R/z)$ with $R$ the radius of the ice patch \DIFdelbegin \DIFdel{. An important }\DIFdelend \DIFaddbegin \DIFadd{and $z$ the depth. A major }\DIFaddend drawback of this method is that additional field observations of $N$ and $P$ are required to adequately parameterize the model\DIFaddbegin \DIFadd{, }\DIFaddend which makes its use more difficult. To address this concern \DIFaddbegin \DIFadd{(among others)}\DIFaddend , \citet{Laney2017} proposed a semi-empirical parameterization that includes a second exponential coefficient \DIFdelbegin \DIFdel{to equation }\DIFdelend \DIFaddbegin \DIFadd{in Equation }\DIFaddend \ref{eq:edz} to model light decrease \DIFdelbegin \DIFdel{between ice surface and ice-ocean interface.
}\DIFdelend \DIFaddbegin \DIFadd{at the interface between the ice and ocean water at the bottom of the ice layer (Equation \ref{eq:laney2017}):
}\DIFaddend 

\begin{equation}
    \edz{} = \edzero{}\DIFdelbegin \DIFdel{\times }\DIFdelend e\DIFdelbegin \DIFdel{^{-\ked(z)} }\DIFdelend \DIFaddbegin \DIFadd{^{-\ked(z)~z} }\DIFaddend - (\edzero{} - E_d(\text{NS}))\DIFdelbegin \DIFdel{\times }\DIFdelend \DIFaddbegin \DIFadd{~}\DIFaddend e\DIFdelbegin \DIFdel{^{-K_{NS}(z)}
    }\DIFdelend \DIFaddbegin \DIFadd{^{-K_{NS}(z)~z}
    }\DIFaddend \label{eq:laney2017}
\end{equation}

\DIFaddbegin \noindent \DIFaddend where \edzero{} is the irradiance that would be observed under homogeneous snow \DIFdelbegin \DIFdel{/}\DIFdelend \DIFaddbegin \DIFadd{or }\DIFaddend ice cover, $E_d(\text{NS})$ is the irradiance under ice, \DIFaddbegin \DIFadd{and }\DIFaddend $K_{NS}(z)$ describes the \DIFdelbegin \DIFdel{near-surface }\DIFdelend decrease of \edzero{} \DIFdelbegin \DIFdel{.  Both }\DIFdelend \DIFaddbegin \DIFadd{just under the ice layer. Both the }\DIFaddend methods by \citet{Frey2011} and \citet{Laney2017} \DIFdelbegin \DIFdel{allow propagating }\DIFdelend \DIFaddbegin \DIFadd{make it possible to propagate }\DIFaddend local \edz{} vertically under \DIFdelbegin \DIFdel{specific sea icefeatures. Additionally, they in principle allow estimating KEd under homogeneous sea ice. What matters}\DIFdelend \DIFaddbegin \DIFadd{low transmission ice. However, these methods cannot identify and correct for inflated }\ked{} \DIFadd{when profiles are measured beneath an area of high transmission relative to surrounding areas. Additionally}\DIFaddend , when trying to determine primary production by phytoplankton that drift under sea ice and \DIFdelbegin \DIFdel{, therefore , is }\DIFdelend \DIFaddbegin \DIFadd{therefore are }\DIFaddend not static under \DIFdelbegin \DIFdel{some anecdotal sea ice feature, }\DIFdelend \DIFaddbegin \DIFadd{sea ice features, what matters }\DIFaddend is the average shape of the vertical \edz{} profile, which may possibly be predictable using a large-area \meanked{} as under a wavy \DIFdelbegin \DIFdel{open-ocean }\DIFdelend \DIFaddbegin \DIFadd{open ocean }\DIFaddend surface \citep{Zaneveld2001}. 

In this study, using both \DIFdelbegin \DIFdel{in-situ }\DIFdelend \DIFaddbegin \DIFadd{in situ }\DIFaddend data and 3D \DIFdelbegin \DIFdel{Monte-Carlo }\DIFdelend \DIFaddbegin \DIFadd{Monte Carlo }\DIFaddend numerical simulations of radiative transfer, we show that the vertical propagation of average \edz{}, \meanedz{}, is reasonably well approximated by a single exponential decay with a so-called \DIFdelbegin \DIFdel{large-area }\DIFdelend \DIFaddbegin \DIFadd{large area }\ked{}\DIFadd{, }\DIFaddend \meanked{}\DIFdelbegin \DIFdel{under sea-ice covered with }\DIFdelend \DIFaddbegin \DIFadd{, under sea ice covered in }\DIFaddend melt ponds. We further demonstrate that \DIFdelbegin \DIFdel{the large-area }\DIFdelend \meanked{} can be estimated from \DIFdelbegin \DIFdel{measurements of }\DIFdelend the vertical attenuation coefficient for upward radiance \DIFaddbegin \DIFadd{(}\DIFaddend \klu{}\DIFdelbegin \DIFdel{, }\DIFdelend \DIFaddbegin \DIFadd{) }\DIFaddend because the latter is \DIFdelbegin \DIFdel{supposedly }\DIFdelend \DIFaddbegin \DIFadd{believably }\DIFaddend less affected by local surface features of the ice cover\DIFaddbegin \DIFadd{. We implicitly assume that primary production can be adequately modeled using }\meanedz{}\DIFadd{, and we conclude that }\klu{} \DIFadd{is an appropriate AOP for predicting the vertical variations in }\meanedz{} \DIFadd{under sea ice}\DIFaddend .