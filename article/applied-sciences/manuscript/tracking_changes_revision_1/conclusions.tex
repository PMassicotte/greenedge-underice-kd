\section{Conclusions}

Our results show that under spatially heterogeneous sea ice \DIFdelbegin \DIFdel{surface}\DIFdelend \DIFaddbegin \DIFadd{at the surface (and for a homogeneous water column)}\DIFaddend , the average \DIFdelbegin \DIFdel{light }\DIFdelend \DIFaddbegin \DIFadd{irradiance }\DIFaddend profile, \meanedz{}, is well reproduced by a single exponential function. We also showed that propagating \edzero{} using \klu{} is \DIFdelbegin \DIFdel{better a }\DIFdelend \DIFaddbegin \DIFadd{a better }\DIFaddend choice compared to \ked{} under heterogeneous sea ice. Nowadays, radiance measurements are becoming more routinely \DIFdelbegin \DIFdel{measured }\DIFdelend \DIFaddbegin \DIFadd{performed }\DIFaddend during field campaigns\DIFdelbegin \DIFdel{of ecological studies}\DIFdelend , so we argue that one should use \klu{} when available to propagate \edzero{} \DIFaddbegin \DIFadd{through the water column under sea ice}\DIFaddend . The main difficulty remains \DIFdelbegin \DIFdel{at }\DIFdelend \DIFaddbegin \DIFadd{in }\DIFaddend finding good estimates of \DIFaddbegin \DIFadd{averaged }\DIFaddend \edzero{}. In \DIFdelbegin \DIFdel{the last }\DIFdelend \DIFaddbegin \DIFadd{recent }\DIFaddend years, this \DIFdelbegin \DIFdel{became }\DIFdelend \DIFaddbegin \DIFadd{has become }\DIFaddend easier with the development of remotely operated vehicles \citep{Katlein2015, Arndt2017, Nicolaus2013}, remote sensing techniques and drone imagery. In this study\DIFaddbegin \DIFadd{, }\DIFaddend we used a \DIFdelbegin \DIFdel{Monte-Carlo }\DIFdelend \DIFaddbegin \DIFadd{Monte Carlo }\DIFaddend approach to model an idealized surface with a single melt pond (Fig. \DIFdelbegin \DIFdel{4). However, in the Arctic, a complex mosaic composed of ice, snow, leads, melt ponds and open water is characterizing the landscape. Hence, further work using }\DIFdelend \DIFaddbegin \DIFadd{1, Fig. 6). Fig. 11 shows that the effect of a 5 m melt pond is minimized after approximately 20 m. Therefore, when many melt ponds are characterizing an area, if one has to perform a single profile, measuring an upward radiance profile under bare ice as far away as possible from any melt pond would minimize the error in estimating the area-averaged downward irradiance profile using }\klu{}\DIFadd{. Although not representative of a complex Arctic sea ice surface, our simple surface geometry allowed to study the transition from a high to a low transmission sea ice. Further }\DIFaddend 3D \DIFdelbegin \DIFdel{radiative transfer models are needed to fully understand light distribution under spatially heterogeneous surface}\DIFdelend \DIFaddbegin \DIFadd{Monte Carlo work could include a more complex geometry of heterogeneous surfaces}\DIFaddend .