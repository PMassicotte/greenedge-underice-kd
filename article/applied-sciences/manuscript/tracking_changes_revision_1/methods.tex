\section{Material and methods}

\subsection{Study site and field campaign}

The field campaign was part of the GreenEdge project (www.greenedgeproject.info) which was conducted on landfast ice southeast of the Qikiqtarjuaq Island \DIFaddbegin \DIFadd{in the Baffin Bay }\DIFaddend (67.4797N, \DIFdelbegin \DIFdel{-63.7895}\DIFdelend \DIFaddbegin \DIFadd{63.7895}\DIFaddend W). The field operations took place at an ice camp where the water depth was 360 m, from April 20 \DIFdelbegin \DIFdel{until }\DIFdelend \DIFaddbegin \DIFadd{to }\DIFaddend July 27\DIFdelbegin \DIFdel{of }\DIFdelend \DIFaddbegin \DIFadd{, }\DIFaddend 2016 (Supplementary Fig. 1). During the sampling period, the study site experienced changes in the snow cover and \DIFdelbegin \DIFdel{lanfast ice thicknesses thickness between 0.32--49.00 and 105.75--149.31 }\DIFdelend \DIFaddbegin \DIFadd{landfast ice thickness of between 0-49 and 106-149 }\DIFaddend cm, respectively.

\subsection{\DIFdelbegin \DIFdel{Underwater }\DIFdelend \DIFaddbegin \DIFadd{In situ underwater }\DIFaddend light measurements}

\DIFdelbegin \DIFdel{A }\DIFdelend \DIFaddbegin \DIFadd{During the campaign, a }\DIFaddend total of 83 vertical light profiles \DIFdelbegin \DIFdel{using a factory calibrated  }\DIFdelend \DIFaddbegin \DIFadd{were acquired using a factory-calibrated }\DIFaddend ICE-Pro (an ice floe version of the C-OPS\DIFdelbegin \DIFdel{- }\DIFdelend \DIFaddbegin \DIFadd{, or }\DIFaddend Compact-Optical Profiling System\DIFdelbegin \DIFdel{- }\DIFdelend \DIFaddbegin \DIFadd{, }\DIFaddend from Biospherical Instruments Inc.) equipped with both downward \DIFaddbegin \DIFadd{plane }\DIFaddend irradiance \edz{} (\DIFdelbegin \DIFdel{$W~cm^{-2}$}\DIFdelend \DIFaddbegin \wmsquare{}\DIFaddend ) and upward radiance \luz{} (\DIFdelbegin \DIFdel{$W~cm^{-2}~sr^{-1}$) radiometerswere measured during the campaign. The IcePRO }\DIFdelend \DIFaddbegin \wmsquaresr{}\DIFadd{) radiometers. The ICE-Pro }\DIFaddend system is a negatively buoyant instrument with \DIFaddbegin \DIFadd{a cylindrical shape }\DIFaddend 10 inches in diameter \DIFdelbegin \DIFdel{cylindrical shape, }\DIFdelend and is not designed for free-fall casts (as opposed to its \DIFdelbegin \DIFdel{open water }\DIFdelend \DIFaddbegin \DIFadd{open-water }\DIFaddend version). To perform the \DIFdelbegin \DIFdel{triplicate }\DIFdelend profiles, the frame \DIFdelbegin \DIFdel{is manually lowered in }\DIFdelend \DIFaddbegin \DIFadd{was manually lowered into }\DIFaddend an auger hole that \DIFdelbegin \DIFdel{has been cleaned for }\DIFdelend \DIFaddbegin \DIFadd{had been cleaned of }\DIFaddend ice chunks. Once \DIFaddbegin \DIFadd{it was }\DIFaddend underneath the ice layer, \DIFdelbegin \DIFdel{clean and fresh snow is shoveled }\DIFdelend \DIFaddbegin \DIFadd{fresh clean snow was shovelled }\DIFaddend back in the hole \DIFdelbegin \DIFdel{, to prevent any }\DIFdelend \DIFaddbegin \DIFadd{to prevent the creation of a }\DIFaddend bright spot right on top of the sensors\DIFdelbegin \DIFdel{, and great care is }\DIFdelend \DIFaddbegin \DIFadd{. Great care was }\DIFaddend taken not to pollute the hole surroundings (footsteps, water and slush spillage from the auger drilling, etc.). The operator then \DIFdelbegin \DIFdel{steps }\DIFdelend \DIFaddbegin \DIFadd{stepped }\DIFaddend back 50 m, while keeping the sensors right under the ice, to avoid any human shadow on top of the profile. \DIFdelbegin \DIFdel{Then the frame is }\DIFdelend \DIFaddbegin \DIFadd{The frame was then }\DIFaddend lowered manually at a constant descent rate of approximately 0.3 \DIFdelbegin \DIFdel{$m \times s^{-1}$. The above surface }\DIFdelend \DIFaddbegin \DIFadd{m s\textsuperscript{-1}. The above-surface }\DIFaddend atmospheric reference sensor \DIFdelbegin \DIFdel{is }\DIFdelend \DIFaddbegin \DIFadd{was }\DIFaddend fixed on a \DIFaddbegin \DIFadd{steady }\DIFaddend tripod standing on the floe \DIFdelbegin \DIFdel{(very steady), }\DIFdelend approximately 2 m above the surface and above \DIFdelbegin \DIFdel{any neighbour ice camp feature}\DIFdelend \DIFaddbegin \DIFadd{all neighbouring ice camp features}\DIFaddend . Data processing and validation were performed using a protocol inspired by the one proposed by \citet{Smith1984} which is now used by \DIFdelbegin \DIFdel{various }\DIFdelend \DIFaddbegin \DIFadd{the main }\DIFaddend space agencies. Measurements were made at 19 wavelengths: 380, 395, 412, 443, 465, 490, 510, 532, 555, 560, 589, 625, 665, 683, 694, 710, 765, 780 and 875 nm. For this study, \ed{} and \lu{} spectra were interpolated linearly between 400 and 700 nm every 10 nm. \DIFdelbegin \DIFdel{In-situ }\DIFdelend \DIFaddbegin \DIFadd{In situ diffuse }\DIFaddend attenuation coefficients ($K$) for both \ed{} (\ked{}) and \lu{} (\klu{}) were calculated on a 5 \DIFdelbegin \DIFdel{meters }\DIFdelend \DIFaddbegin \DIFadd{m }\DIFaddend sliding window (10--15 m, 15--20 m, $\ldots$, 70--75 m, 75--80 m) starting at 10 \DIFdelbegin \DIFdel{meters }\DIFdelend \DIFaddbegin \DIFadd{m }\DIFaddend to reduce the effects of surface heterogeneity. A total of 72 044 non-linear models were calculated to estimate \DIFaddbegin \DIFadd{both }\DIFaddend $K$ \DIFdelbegin \DIFdel{from equation }\DIFdelend \DIFaddbegin \DIFadd{coefficients from Equation }\DIFaddend 1 (83 profiles $\times$ 14 depths $\times$ 31 wavelengths $\times$ 2 \DIFdelbegin \DIFdel{lights }\DIFdelend \DIFaddbegin \DIFadd{radiometric quantities }\DIFaddend (\ed{}, \lu{})). A conservative \DIFdelbegin \DIFdel{$R^2$ }\DIFdelend \DIFaddbegin \rsquared{} \DIFaddend of 0.99 was used \DIFaddbegin \DIFadd{essentially }\DIFaddend to filter out \DIFdelbegin \DIFdel{poor models (i.e. noisy profilesthat were not following an exponential decrease)}\DIFdelend \DIFaddbegin \DIFadd{noisy profiles}\DIFaddend . 42 407 models were kept for subsequent analysis.

\subsection{3D \DIFdelbegin \DIFdel{Monte-Carlo }\DIFdelend \DIFaddbegin \DIFadd{Monte Carlo }\DIFaddend numerical simulations}

\subsubsection{Theory and geometry}

3D numerical \DIFdelbegin \DIFdel{Monte-Carlo }\DIFdelend \DIFaddbegin \DIFadd{Monte Carlo }\DIFaddend simulation is a convenient approach \DIFdelbegin \DIFdel{to model }\DIFdelend \DIFaddbegin \DIFadd{for modelling }\DIFaddend the light field under spatially heterogeneous sea \DIFaddbegin \DIFadd{surfaces }\DIFaddend \citep{Mobley_ocean_optics_book, Petrich2012, Katlein2014, Katlein2016}. They are simple to understand \DIFdelbegin \DIFdel{, }\DIFdelend \DIFaddbegin \DIFadd{and }\DIFaddend versatile, and incident light, \DIFdelbegin \DIFdel{inherent optical properties (IOPs ) }\DIFdelend \DIFaddbegin \DIFadd{IOPs }\DIFaddend and geometry can be easily changed. In this study, we used SimulO, a \DIFdelbegin \DIFdel{Monte-Carlo software that allows simulating }\DIFdelend \DIFaddbegin \DIFadd{3D Monte Carlo software program that simulates }\DIFaddend the propagation of light in \DIFdelbegin \DIFdel{various geometries from optical instruments to open or ice covered oceanic }\DIFdelend \DIFaddbegin \DIFadd{optical instruments or in ocean }\DIFaddend waters \citep{Leymarie2010}. \DIFaddbegin \DIFadd{Our objective was to simulate the propagation of sunlight underneath heterogeneous ice-covered ocean waters. }\DIFaddend Simulations were performed in an idealized ocean described by a cylinder of 120 m radius and 150 m depth \DIFdelbegin \DIFdel{. Given our interest in surface light profiles, the deepest software photons counter was placed at 25 m depth}\DIFdelend \DIFaddbegin \DIFadd{(Fig. 1)}\DIFaddend . The water IOPs were selected to reflect pre-bloom conditions \DIFdelbegin \DIFdel{(a = b = 0.05 $m^{-1}$) }\DIFdelend in the green/blue spectral region \DIFaddbegin \DIFadd{(a = b = 0.05 }\mminus{}\DIFadd{)}\DIFaddend . These typical averaged values were measured \DIFdelbegin \DIFdel{in the visible range }\DIFdelend during the GreenEdge 2016 campaign using an \DIFdelbegin \DIFdel{in-situ spectrophotometer (ACS, }\DIFdelend \DIFaddbegin \DIFadd{in situ spectrophotometer (ac-s from }\DIFaddend Sea-Bird Scientific) and represent the contribution of both pure water and \DIFdelbegin \DIFdel{water’s }\DIFdelend \DIFaddbegin \DIFadd{the water }\DIFaddend constituents. The scattering phase function was described by a Fournier-Forand analytic form with a 3\% backscatter fraction \citep{Fournier1994, Mobley2002}. \DIFdelbegin \DIFdel{Sea }\DIFdelend \DIFaddbegin \DIFadd{The inclusion of a 3D sea ice layer at the upper boundary of the ocean would require extensive computing power because of the high scattering properties of sea ice. Instead, sea }\DIFaddend ice was incorporated at the upper boundary of the ocean using a 2D \DIFdelbegin \DIFdel{emitting surface }\DIFdelend \DIFaddbegin \DIFadd{light-emitting surface with a radius }\DIFaddend of 100 m\DIFdelbegin \DIFdel{radius. A 5 m radius melt pond was set-up at the center of the surface (Fig. 1). The photon emission of the melt pond surface has four times the intensity of the surrounding ice which corresponds to typical conditions found in Arctic during summer \mbox{%DIFAUXCMD
\citep{Perovich2016}}\hspace{0pt}%DIFAUXCMD
}\DIFdelend \DIFaddbegin \DIFadd{. The angular distribution and amplitude of the light field emitted by the surface was chosen to mimic observed field data \mbox{%DIFAUXCMD
\citep{Girard2018}}\hspace{0pt}%DIFAUXCMD
}\DIFaddend . SimulO does not allow \DIFdelbegin \DIFdel{to use arbitrary  emission angular distribution }\DIFdelend \DIFaddbegin \DIFadd{the use of arbitrary angular distribution for photon-emitting surfaces}\DIFaddend . To overcome this problem, two \DIFdelbegin \DIFdel{lambertian sources of 90 and 60 degrees }\DIFdelend \DIFaddbegin \DIFadd{sources of photons }\DIFaddend were summed up in order to \DIFdelbegin \DIFdel{mimic observed under ice radiance light field \mbox{%DIFAUXCMD
\citep{Girard2018} }\hspace{0pt}%DIFAUXCMD
and reproduce the subsurface light maximums observed between $\approx$ 5--20 meters (supplementary Fig. 4}\DIFdelend \DIFaddbegin \DIFadd{reproduce an observed under-ice light field (Fig. 2}\DIFaddend ). The \DIFdelbegin \DIFdel{same emission angular distribution was used for both ice and melt pond surfaces. For this purpose of this study, the small difference between the light field shape measured under melt pond vs ice is much less important compared the their difference of intensity \mbox{%DIFAUXCMD
\citep{Girard2018}}\hspace{0pt}%DIFAUXCMD
}\DIFdelend \DIFaddbegin \DIFadd{first source was a regular Lambertian emitting surface while the second was a Lambertian emitting surface but restricted to an emission within 60 degrees of the zenith angle. A 5-m radius melt pond was set up at the centre of the emitting surface (Fig. 1). The melt pond had the same emitting angular distribution as the surrounding ice. Its intensity was four times higher than the surrounding ice, which corresponds to typical conditions found in the Arctic during summer \mbox{%DIFAUXCMD
\citep{Perovich2016}}\hspace{0pt}%DIFAUXCMD
}\DIFaddend .

\DIFaddbegin \begin{figure}[H]
	\centering
	\includegraphics[scale = 1]{../../../../graphs/fig1.pdf}
	\caption{\DIFaddFL{Spatial configuration used for the 3D Monte Carlo numerical simulations. (}\textbf{\DIFaddFL{A}}\DIFaddFL{) Surface view showing the percentage of the total area covered by the melt pond over the areas described by the black lines. For each of these areas, light profiles were averaged (see Fig. 7). For visualization purpose, lines of the horizontal sampling distances have been plotted only at 5 m intervals. (}\textbf{\DIFaddFL{B}}\DIFaddFL{) 2D side view showing the 3D volume for which simulated data were extracted and how photon detectors were placed in the water column. Orange arrows schematize incident light sources.}}
\end{figure}

\DIFadd{Given our interest in surface light profiles, }\DIFaddend 2D horizontal \DIFaddbegin \DIFadd{software }\DIFaddend detectors were placed vertically every 0.5 \DIFdelbegin \DIFdel{meters, up to }\DIFdelend \DIFaddbegin \DIFadd{m, from 0.5 m up to a depth of }\DIFaddend 25 \DIFdelbegin \DIFdel{meters}\DIFdelend \DIFaddbegin \DIFadd{m}\DIFaddend . Detectors include \DIFdelbegin \DIFdel{1-$m^2$ pixels measuring planar irradiance for the downward face and radiance for the upward face }\DIFdelend \DIFaddbegin \DIFadd{1 m\textsuperscript{2} pixels measuring downward irradiance and upward radiance }\DIFaddend (5 \DIFdelbegin \DIFdel{degrees }\DIFdelend \DIFaddbegin \DIFadd{degree }\DIFaddend half angle). In order to avoid the effect of the boundary (i.e. absorption by the side of the cylinder used to simulate the water column), data outside a radius of 50 \DIFdelbegin \DIFdel{meters }\DIFdelend \DIFaddbegin \DIFadd{m }\DIFaddend were not used \DIFaddbegin \DIFadd{(see the green box in Fig. 1)}\DIFaddend . A total number of \DIFdelbegin \DIFdel{7.14e10 }\DIFdelend \DIFaddbegin \DIFadd{$7.14 \times 10^{10}$ }\DIFaddend photons were simulated \DIFdelbegin \DIFdel{in order the obtained sufficient }\DIFdelend \DIFaddbegin \DIFadd{to obtain a sufficient number of }\DIFaddend upwelling photons. \DIFaddbegin \DIFadd{The simulation took approximately 6 000 hours distributed over 2 000 CPU cores. Since the geometry was symmetrical azimuthally, irradiance and radiance were averaged over the azimuth in order to raise the signal-to-noise ratio. }\DIFaddend Due to the low scattering coefficients used to reproduce \DIFdelbegin \DIFdel{in-situ }\DIFdelend \DIFaddbegin \DIFadd{in situ }\DIFaddend conditions observed during the sampling campaign, radiance profiles were noisy because \DIFdelbegin \DIFdel{only }\DIFdelend a small number of upward photons could be captured. To address this issue, radiance profiles were smoothed \DIFdelbegin \DIFdel{out using Gaussian fittings (supplementary Fig. 5). 
The simulation took approximately 6000 hours distributed over 2000 CPU cores. 
}\DIFdelend \DIFaddbegin \DIFadd{using a Gaussian fit (Supplementary Fig. 2). 
}\DIFaddend 

\DIFdelbegin \subsubsection{\DIFdel{Estimation of different reference light profiles}}
%DIFAUXCMD
\addtocounter{subsubsection}{-1}%DIFAUXCMD
\DIFdelend \DIFaddbegin \begin{figure}[H]
	\centering
	\includegraphics[scale = 1]{../../../../graphs/fig2.pdf}
	\caption{\DIFaddFL{Comparison of the under-ice measured downward radiance distribution (the average cosine is $\approx$ 0.61, \mbox{%DIFAUXCMD
\cite{Girard2018}}\hspace{0pt}%DIFAUXCMD
) and the emitting source angular distribution used in the paper.}}
\end{figure}
\DIFaddend 

\DIFdelbegin \DIFdel{Using the Monte-Carlo simulation, data were averaged accordingly to six different radius with therefore varying melt pond proportions to explore how melt pond influence }\DIFdelend \DIFaddbegin \subsubsection{\DIFadd{Estimation of reference and local light profiles}}

\DIFadd{To explore how the melt pond influences }\DIFaddend the averaged underwater irradiance and radiance profiles (Fig. 1)\DIFdelbegin \DIFdel{. This is equivalent }\DIFdelend \DIFaddbegin \DIFadd{, data from the Monte Carlo simulation were averaged according to six different radii, corresponding }\DIFaddend to varying melt pond \DIFdelbegin \DIFdel{concentration. For each case, }\DIFdelend \DIFaddbegin \DIFadd{spatial proportions. The }\DIFaddend simulated light profiles were averaged within the following surface areas: (1) 10 \DIFdelbegin \DIFdel{meters }\DIFdelend \DIFaddbegin \DIFadd{m }\DIFaddend radius (25\% melt pond cover), (2) 11.18 \DIFdelbegin \DIFdel{meters }\DIFdelend \DIFaddbegin \DIFadd{m }\DIFaddend radius (20\% melt pond cover), (3) 12.91 \DIFdelbegin \DIFdel{meters }\DIFdelend \DIFaddbegin \DIFadd{m }\DIFaddend radius (15\% melt pond cover), (4) \DIFdelbegin \DIFdel{15.811 meters }\DIFdelend \DIFaddbegin \DIFadd{15.81 m }\DIFaddend radius (10\% melt pond cover), (5) \DIFdelbegin \DIFdel{22.361 meters }\DIFdelend \DIFaddbegin \DIFadd{22.36 m }\DIFaddend radius (5\% melt pond cover) and (6) 50 \DIFdelbegin \DIFdel{meters }\DIFdelend \DIFaddbegin \DIFadd{m }\DIFaddend radius (1\% melt pond cover). For each of these \DIFdelbegin \DIFdel{configurations, }\DIFdelend \DIFaddbegin \DIFadd{six configurations, the corresponding }\DIFaddend averaged light profile, \meanedz{}, was subsequently viewed as an adequate description of the average underwater light field. \DIFdelbegin \DIFdel{A total of 45 light profiles}\DIFdelend \DIFaddbegin \DIFadd{For the remainder of the text, these averaged profiles are referred to as reference light profiles. Furthermore, 50 light profiles, }\DIFaddend evenly spaced by \DIFdelbegin \DIFdel{one meter around }\DIFdelend \DIFaddbegin \DIFadd{1 m from }\DIFaddend the melt pond \DIFdelbegin \DIFdel{were further }\DIFdelend \DIFaddbegin \DIFadd{centre, were }\DIFaddend extracted to mimic local measurements of light and to calculate associated \DIFdelbegin \DIFdel{attenuation coefficients (colored circles in Fig. 1)}\DIFdelend \DIFaddbegin \DIFadd{diffuse attenuation coefficients}\DIFaddend .

\subsection{Statistical analysis}

All statistical \DIFdelbegin \DIFdel{analysis }\DIFdelend \DIFaddbegin \DIFadd{analyses }\DIFaddend and graphics were carried out with R 3.5.1 \citep{RCoreTeam2018}. 