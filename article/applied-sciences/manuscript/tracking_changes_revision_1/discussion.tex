\section{Discussion}

In the Arctic, melt pond coverage, lead coverage\DIFaddbegin \DIFadd{, and ice }\DIFaddend and \DIFdelbegin \DIFdel{ice/}\DIFdelend snow thickness can vary \DIFdelbegin \DIFdel{highly }\DIFdelend \DIFaddbegin \DIFadd{greatly }\DIFaddend in both time and space \DIFdelbegin \DIFdel{\mbox{%DIFAUXCMD
\citep{Landy2014, Eicken2004}}\hspace{0pt}%DIFAUXCMD
}\DIFdelend \DIFaddbegin \DIFadd{\mbox{%DIFAUXCMD
\citep{Landy2014,Eicken2004}}\hspace{0pt}%DIFAUXCMD
}\DIFaddend . Due to this sea ice heterogeneity, local \DIFdelbegin \DIFdel{under ice }\DIFdelend \DIFaddbegin \DIFadd{under-ice }\DIFaddend measurements of downward irradiance are \DIFdelbegin \DIFdel{often }\DIFdelend \DIFaddbegin \DIFadd{sometimes }\DIFaddend characterized by subsurface light \DIFdelbegin \DIFdel{maximums }\DIFdelend \DIFaddbegin \DIFadd{maxima }\DIFaddend (Fig. \DIFdelbegin \DIFdel{2}\DIFdelend \DIFaddbegin \DIFadd{3}\DIFaddend ). To model such profiles, \citet{Laney2017} proposed a semi-empirical parameterization using two exponential terms (see \DIFdelbegin \DIFdel{equation }\DIFdelend \DIFaddbegin \DIFadd{Equation }\DIFaddend 3). Whereas their method might provide adequate estimations of instantaneous downward \DIFaddbegin \DIFadd{diffuse }\DIFaddend attenuation coefficients at specific locations, fitting a double exponential might not be ideal because data \DIFdelbegin \DIFdel{is }\DIFdelend \DIFaddbegin \DIFadd{are }\DIFaddend modelled locally and do not provide an adequate description of the average light field (\meanedz{}) as it would be seen, for example, by drifting phytoplankton cells. In such conditions\DIFdelbegin \DIFdel{it was argued that under ice, }\DIFdelend \DIFaddbegin \DIFadd{, this paper argues that under-ice }\DIFaddend irradiance measurements should be analyzed in the context of ice and surface properties within a radius of several \DIFdelbegin \DIFdel{meters }\DIFdelend \DIFaddbegin \DIFadd{metres over the horizontal distance }\DIFaddend since local measurements \DIFdelbegin \DIFdel{do not reproduce the full variability of the under ice light field\mbox{%DIFAUXCMD
\citep{Katlein2015}}\hspace{0pt}%DIFAUXCMD
}\DIFdelend \DIFaddbegin \DIFadd{cannot be used as a proxy of the average light field}\DIFaddend .

Using \DIFdelbegin \DIFdel{in-situ }\DIFdelend \DIFaddbegin \DIFadd{in situ }\DIFaddend light measurements, it was found that \ed{} and \lu{} (and therefore \ked{} and \klu{}) were highly correlated \DIFdelbegin \DIFdel{bellow }\DIFdelend \DIFaddbegin \DIFadd{below }\DIFaddend 10 \DIFdelbegin \DIFdel{meters }\DIFdelend \DIFaddbegin \DIFadd{m depth }\DIFaddend (Fig. \DIFdelbegin \DIFdel{3}\DIFdelend \DIFaddbegin \DIFadd{4}\DIFaddend , Fig. \DIFdelbegin \DIFdel{4}\DIFdelend \DIFaddbegin \DIFadd{5}\DIFaddend ), even when subsurface light maxima were present (Fig. \DIFdelbegin \DIFdel{2). One possible explanation is that a }%DIFDELCMD < \lu{} %%%
\DIFdel{radiometer measures scattered light originating from a larger surface area, which reduce the effect of sea ice heterogeneity. Accordingly}\DIFdelend \DIFaddbegin \DIFadd{3). Furthermore}\DIFaddend , no subsurface light maxima were observed in the \DIFdelbegin \DIFdel{in-situ }\DIFdelend \DIFaddbegin \DIFadd{in situ upward }\DIFaddend radiance profiles. \DIFdelbegin \DIFdel{This reinforce }\DIFdelend \DIFaddbegin \DIFadd{The reason is that a }\lu{} \DIFadd{radiometer measures upwelling photons coming from deeper depth that have undergone more scattering.  These photons thus originate from a larger surface area. This reinforces }\DIFaddend the idea that \lu{} is less influenced by \DIFdelbegin \DIFdel{sea-ice }\DIFdelend \DIFaddbegin \DIFadd{sea ice }\DIFaddend surface heterogeneity. 

Based on \DIFdelbegin \DIFdel{Monte-Carlo }\DIFdelend \DIFaddbegin \DIFadd{Monte Carlo }\DIFaddend simulations, our results showed that the average downward \DIFdelbegin \DIFdel{light profile, }%DIFDELCMD < \meanedz%%%
\DIFdelend \DIFaddbegin \DIFadd{irradiance profile, }\meanedz{}\DIFaddend , under heterogeneous sea ice cover follows a \DIFdelbegin \DIFdel{single term }\DIFdelend \DIFaddbegin \DIFadd{single-term }\DIFaddend exponential function, even when melt ponds occupy a large fraction of the study area (Fig. \DIFdelbegin \DIFdel{6}\DIFdelend \DIFaddbegin \DIFadd{7}\DIFaddend ). This is similar to what is observed under a wavy ice-free surface \citep{Zaneveld2001}. However, estimating \DIFdelbegin %DIFDELCMD < \edz{} %%%
\DIFdelend \DIFaddbegin \meanedz{} \DIFaddend for a given area is not straightforward\DIFaddbegin \DIFadd{, }\DIFaddend as it requires a large number of local profiles under the sea ice. An intuitive \DIFdelbegin \DIFdel{workaround to derive }\DIFdelend \DIFaddbegin \DIFadd{alternative to deriving the }\DIFaddend attenuation coefficient is to use upward radiance\DIFaddbegin \DIFadd{, }\DIFaddend which is less influenced by sea surface heterogeneity compared to downward irradiance (Fig. \DIFdelbegin \DIFdel{2}\DIFdelend \DIFaddbegin \DIFadd{3}\DIFaddend , Fig. \DIFdelbegin \DIFdel{3}\DIFdelend \DIFaddbegin \DIFadd{4}\DIFaddend , Fig. \DIFdelbegin \DIFdel{4). Monte-Carlo }\DIFdelend \DIFaddbegin \DIFadd{5). Monte Carlo }\DIFaddend simulations showed that a local estimation of \klu{} \DIFdelbegin \DIFdel{could be }\DIFdelend \DIFaddbegin \DIFadd{was }\DIFaddend a good proxy for \meanked{} \DIFdelbegin \DIFdel{. Accordingly, our results showed that propagating under sea ice average irradiance (}%DIFDELCMD < \edzero{}%%%
\DIFdel{) }\DIFdelend \DIFaddbegin \DIFadd{and that }\DIFaddend using \klu{} rather than \ked{} provided better estimations of the average downward profile \DIFaddbegin \DIFadd{by reducing the average error by approximately a factor of two }\DIFaddend (Fig. \DIFdelbegin \DIFdel{8, Fig. 9}\DIFdelend \DIFaddbegin \DIFadd{11}\DIFaddend ). 

There are at least two main factors influencing the quality of \DIFdelbegin \DIFdel{in-situ downward }\DIFdelend \DIFaddbegin \DIFadd{in situ downward irradiance }\DIFaddend measurements under heterogeneous sea ice. The first factor is the horizontal distance from the \DIFdelbegin \DIFdel{melt pond ridge}\DIFdelend \DIFaddbegin \DIFadd{centre of the melt pond}\DIFaddend . Although the relative error of propagating \edzero{} using both \ked{} and \klu{} showed the same pattern, the largest error occurred when using local estimations of \ked{} \DIFdelbegin \DIFdel{made between 1 and }\DIFdelend \DIFaddbegin \DIFadd{directly below the melt pond and up to }\DIFaddend 10 \DIFdelbegin \DIFdel{meters outside }\DIFdelend \DIFaddbegin \DIFadd{m from }\DIFaddend the melt pond \DIFaddbegin \DIFadd{edge }\DIFaddend (Fig. \DIFdelbegin \DIFdel{9}\DIFdelend \DIFaddbegin \DIFadd{11}\DIFaddend ). In contrast, \DIFdelbegin \DIFdel{in the vicinity of the melt pond, the relative errors associated to }\DIFdelend \DIFaddbegin \DIFadd{the relative error associated with }\DIFaddend the use of \klu{} was much lower and stabilized just after approximately \DIFdelbegin \DIFdel{5 meters}\DIFdelend \DIFaddbegin \DIFadd{10 m from the centre of the melt pond}\DIFaddend . The second factor driving the relative error of local measurements is the proportion occupied by melt ponds over the area of interest (Fig. \DIFdelbegin \DIFdel{9}\DIFdelend \DIFaddbegin \DIFadd{11}\DIFaddend ). Indeed, higher proportions of melt pond \DIFdelbegin \DIFdel{allows }\DIFdelend \DIFaddbegin \DIFadd{allow }\DIFaddend for more light to penetrate in the water column. Hence, local measurements made under surrounding ice are more likely to show subsurface light maxima (see \citet{Frey2011}). Accordingly, when melt ponds accounted for 1\% of the total area, averaged \DIFdelbegin \DIFdel{errors }\DIFdelend \DIFaddbegin \DIFadd{error }\DIFaddend in \edz{} using \klu{} was 1.33\% but increased to 18\% when the melt pond occupied 25\% of the total area (Fig. \DIFdelbegin \DIFdel{9}\DIFdelend \DIFaddbegin \DIFadd{11}\DIFaddend ).