\section{Results}

\subsection{Comparing in situ \DIFaddbegin \DIFadd{downward }\DIFaddend irradiance (\ed{}) and \DIFaddbegin \DIFadd{upward }\DIFaddend radiance (\lu{}) measurements}

An example showing in situ downward irradiance (\ed{}) profiles and upward radiance (\lu{}) profiles at 16 \DIFdelbegin \DIFdel{different }\DIFdelend visible wavelengths measured under ice is presented in Fig. \DIFdelbegin \DIFdel{2. }\DIFdelend \DIFaddbegin \DIFadd{3. }\DIFaddend For the \ed{} profiles, subsurface light maxima at \DIFaddbegin \DIFadd{a depth of }\DIFaddend around 10 \DIFdelbegin \DIFdel{meters }\DIFdelend \DIFaddbegin \DIFadd{m }\DIFaddend are clearly visible between 400 and 560 nm. These peaks are not visible in the yellow/red \DIFdelbegin \DIFdel{regions }\DIFdelend \DIFaddbegin \DIFadd{region }\DIFaddend (580--700 nm). For the \lu{} profiles, no subsurface light maxima were found at any wavelength. To \DIFdelbegin \DIFdel{look closer }\DIFdelend \DIFaddbegin \DIFadd{have a closer look }\DIFaddend at the shape of both \ed{} and \lu{} light profiles, data below 10 m \DIFdelbegin \DIFdel{have been }\DIFdelend \DIFaddbegin \DIFadd{were }\DIFaddend normalized to the value at 10 m (Fig. \DIFdelbegin \DIFdel{3}\DIFdelend \DIFaddbegin \DIFadd{4}\DIFaddend ). Below 10 m and between 400 and 580 nm, both \ed{} and \lu{} profiles \DIFdelbegin \DIFdel{follow the same pattern }\DIFdelend \DIFaddbegin \DIFadd{presented the same shape (i.e. yield the same rate of extinction }\DIFaddend with increasing depth\DIFdelbegin \DIFdel{. At larger }\DIFdelend \DIFaddbegin \DIFadd{). At longer }\DIFaddend wavelengths ($\ge$ 600 nm), differences between the shapes of \ed{} and \lu{} profiles \DIFdelbegin \DIFdel{increase. A linear regression analysis between all in situ normalized }%DIFDELCMD < \ed{} %%%
\DIFdel{and }%DIFDELCMD < \lu{} %%%
\DIFdel{profiles shows that determination coefficients ($R^2$) range between 0.75 and 1 (Supplementary Fig. 3). A sharp decrease and a high variability of calculated $R^2$ occurred beyond 575 nm. This suggests a gradual decoupling between }%DIFDELCMD < \ed{} %%%
\DIFdel{and }%DIFDELCMD < \lu{} %%%
\DIFdel{profiles at larger wavelengths, likely due to the effect of inelastic scattering (mostly, Raman). }%DIFDELCMD < 

%DIFDELCMD < %%%
\DIFdelend \DIFaddbegin \DIFadd{increased. }\DIFaddend Irradiance and radiance \DIFaddbegin \DIFadd{diffuse }\DIFaddend attenuation coefficients (\ked{} and \DIFdelbegin %DIFDELCMD < \klu%%%
\DIFdelend \DIFaddbegin \klu{}\DIFaddend ) calculated on \DIFdelbegin \DIFdel{five meters depth layers }\DIFdelend \DIFaddbegin \DIFadd{layers of a 5 m depth }\DIFaddend are compared in Fig. \DIFdelbegin \DIFdel{4 }\DIFdelend \DIFaddbegin \DIFadd{5 }\DIFaddend for all 83 profiles. In the blue/green/yellow regions (400--580 nm), the determination coefficients between \DIFdelbegin %DIFDELCMD < \ked{} %%%
\DIFdel{and }%DIFDELCMD < \klu %%%
\DIFdelend \DIFaddbegin \klu{} \DIFadd{and }\ked{} \DIFaddend varied between 0.98 at the surface (10--15 m) and 0.64 at depth (\DIFdelbegin \DIFdel{75--80 }\DIFdelend \DIFaddbegin \DIFadd{75-80 }\DIFaddend m). For most of the surface \DIFdelbegin \DIFdel{layer}\DIFdelend \DIFaddbegin \DIFadd{layers}\DIFaddend , regression lines \DIFdelbegin \DIFdel{lined-up }\DIFdelend \DIFaddbegin \DIFadd{lined up }\DIFaddend with the 1:1 lines. Slight deviations from the 1:1 lines started to appear after 60 \DIFdelbegin \DIFdel{meters }\DIFdelend \DIFaddbegin \DIFadd{m }\DIFaddend where \ked{} was on average higher than \klu{}. \DIFdelbegin \DIFdel{Relationships }\DIFdelend \DIFaddbegin \DIFadd{The relationships }\DIFaddend including orange and red wavelengths are presented in Supplementary Fig. 3. \DIFdelbegin %DIFDELCMD < 

%DIFDELCMD < %%%
\subsection{\DIFdel{3D Monte-Carlo numerical simulations}}
%DIFAUXCMD
\addtocounter{subsection}{-1}%DIFAUXCMD
%DIFDELCMD < 

%DIFDELCMD < %%%
\subsubsection{\DIFdel{Simulated irradiance and radiance}}
%DIFAUXCMD
\addtocounter{subsubsection}{-1}%DIFAUXCMD
%DIFDELCMD < 

%DIFDELCMD < %%%
\DIFdel{Under the melt pond, the relative density of irradiance photons was higher than that of the radiance photons (Fig. 5). Another key difference is that simulated radiance was more diffuse compared to irradiance. Over the simulated study area (50 meters radius, Fig. 1), a total of 403 irradianceprofiles with and without subsurface light maxima were extracted (Fig. 6).  For these profiles, the number of "captured" photons for }\DIFdelend \DIFaddbegin \DIFadd{A linear regression analysis between all in situ normalized }\DIFaddend \ed{} \DIFdelbegin \DIFdel{varied between $4.8 \times 10^5$ }\DIFdelend and \DIFdelbegin \DIFdel{$12 \times 10^6$. Due to low scattering, the number of upward photons was much lower and ranged between 25 }\DIFdelend \DIFaddbegin \lu{} \DIFadd{profiles showed that determination coefficients (}\rsquared{}\DIFadd{) range between 0.75 }\DIFaddend and \DIFdelbegin \DIFdel{400 (Fig. }\DIFdelend \DIFaddbegin \DIFadd{1 (Supplementary Fig. 4). A sharp decrease and a high variability of calculated }\rsquared{} \DIFadd{occurred beyond 575 nm. This suggests a gradual decoupling between }\ed{} \DIFadd{and }\lu{} \DIFadd{profiles at longer wavelengths, likely due to the effect of inelastic scattering (mostly, Raman). 
}

\subsection{\DIFadd{3D Monte Carlo numerical simulations}}

\DIFadd{Fig. }\DIFaddend 6 \DIFdelbegin \DIFdel{). Averaged irradiance, (}%DIFDELCMD < \meanedz{}%%%
\DIFdel{), and }\DIFdelend \DIFaddbegin \DIFadd{shows cross-sections of the simulated downward irradiance and upward radiance. A key difference for the upcoming discussion is that the simulated upward }\DIFaddend radiance \DIFdelbegin \DIFdel{, (}%DIFDELCMD < \meanluz{}%%%
\DIFdel{), reference profilesshowed the same pattern where the number of photons increased with increasing melt pond coverage (Fig. 6)}\DIFdelend \DIFaddbegin \DIFadd{was more homogeneous compared to the simulated downward irradiance. Fig. 7 shows the reference irradiance, }\edz{}\DIFadd{, and reference radiance, }\luz{}\DIFadd{, profiles}\DIFaddend . The highest \DIFdelbegin \DIFdel{density of photons occured }\DIFdelend \DIFaddbegin \DIFadd{irradiance and radiance occurred }\DIFaddend when the melt pond occupied 25\% of the sampling area, allowing for more light to propagate in the water column. \DIFdelbegin \DIFdel{Note that none of the averaged }%DIFDELCMD < \meanedz{} %%%
\DIFdel{and }%DIFDELCMD < \meanluz{} %%%
\DIFdelend \DIFaddbegin \DIFadd{None of the }\edz{} \DIFadd{and }\luz{} \DIFaddend reference profiles showed subsurface light maxima\DIFdelbegin \DIFdel{(Fig. 6). }%DIFDELCMD < 

%DIFDELCMD < %%%
\subsubsection{\DIFdel{Attenuation coefficients and propagating light profiles}}
%DIFAUXCMD
\addtocounter{subsubsection}{-1}%DIFAUXCMD
%DIFDELCMD < 

%DIFDELCMD < %%%
\DIFdel{Some irradiance profiles extracted outside }\DIFdelend \DIFaddbegin \DIFadd{. Fig. 8 shows the 50 simulated local downward irradiance and upward radiance light profiles evenly spaced by 1 m from }\DIFaddend the melt pond \DIFdelbegin \DIFdel{at distances between 5 and 15 meters of the center (to mimic local radiometric measurements) showed the same subsurface light maxima (Fig. 7) as observed on in situ profiles (Fig. 2) . Beyond approximately 15 meters, }\DIFdelend \DIFaddbegin \DIFadd{centre. Local downward irradiance profiles under the melt pond (0--5 m) showed a rapid decrease with increasing depth described by a monotonically exponential or quasi-exponential decrease. Local simulated downward irradiance profiles just outside the melt pond (5--10 m) were characterized with }\DIFaddend subsurface light maxima \DIFdelbegin \DIFdel{disappeared and }\DIFdelend \DIFaddbegin \DIFadd{occurring at a depth of between approximately 5 and 10 m. Further away from the melt pond centre, downward }\DIFaddend irradiance profiles followed a monotonically exponential or quasi-exponential decrease\DIFdelbegin \DIFdel{(equation 1). Note that subsurface maxima were not found on radiance profiles as with in situ data. From both }\DIFdelend \DIFaddbegin \DIFadd{. None of the simulated upward radiance light profiles presented subsurface light maxima (Fig. 8). From local simulated }\DIFaddend irradiance and radiance profiles \DIFaddbegin \DIFadd{(Fig. 8)}\DIFaddend , \ked{} \DIFaddbegin \DIFadd{and }\klu{} \DIFadd{were calculated by fitting Equation 1 between 0 and 25 m. Results are presented in Fig. 9. }\ked{} \DIFaddend varied between 0.065 and \DIFdelbegin \DIFdel{0.096 $m^{-1}$ }\DIFdelend \DIFaddbegin \DIFadd{0.157 }\mminus{} \DIFaddend and \klu{} between 0.079 and \DIFdelbegin \DIFdel{0.1 $m^{-1}$ (Fig. 7, supplementary Fig. 6). }%DIFDELCMD < 

%DIFDELCMD < %%%
\DIFdel{Propagating surface reference light (}%DIFDELCMD < \edzero{}%%%
\DIFdel{, surface values of the colored }\DIFdelend \DIFaddbegin \DIFadd{0.116 }\mminus{}\DIFadd{. These }\ked{} \DIFadd{and }\klu{} \DIFadd{were used to propagate light downward from surface reference values }\edzero{}\DIFadd{. Fig. 10 shows the profiles resulting from this operation. A greater dispersion around the reference profiles (thick black }\DIFaddend lines in Fig. \DIFdelbegin \DIFdel{6) through the water column }\DIFdelend \DIFaddbegin \DIFadd{10) occurred when }\DIFaddend using \ked{} \DIFdelbegin \DIFdel{resulted in a greater variability }\DIFdelend compared to the profiles generated with \DIFaddbegin \DIFadd{similarly derived }\DIFaddend \klu{} \DIFdelbegin \DIFdel{(Fig. 8)}\DIFdelend \DIFaddbegin \DIFadd{values}\DIFaddend . The relative differences between \DIFdelbegin \DIFdel{both }\DIFdelend \DIFaddbegin \DIFadd{the }\DIFaddend depth-integrated \DIFdelbegin \DIFdel{(i.e. total number of photons) reference profiles and predicted profiles }\DIFdelend \DIFaddbegin \DIFadd{values of each local profiles (coloured lines in Fig. 10) and the depth-integrated values of the reference profiles (thick black lines in Fig. 10) }\DIFaddend were used to quantify the error of using either \ked{} or \DIFdelbegin %DIFDELCMD < \klu %%%
\DIFdelend \DIFaddbegin \klu{} \DIFaddend as a proxy to predict downward irradiance in the water column (Fig. \DIFdelbegin \DIFdel{9). Overall, the greatest errors on predictions reached approximately 38\% when using }\DIFdelend \DIFaddbegin \DIFadd{11). Below the melt pond, }\DIFaddend \ked{} \DIFdelbegin \DIFdel{at approximately }\DIFdelend \DIFaddbegin \DIFadd{overestimated the total downward irradiance by up to 40\% for the 1\% melt pond reference surface. In this region, the local $K$ coefficients are inflated. In the transition region, between }\DIFaddend 5 \DIFdelbegin \DIFdel{meters from ice ridge }\DIFdelend \DIFaddbegin \DIFadd{and 10 m from the centre of the melt pond, where subsurface maxima are observed, }\ked{} \DIFadd{underestimated the downward irradiance by up to 35\% for the 25\% melt pond reference surface. Further away from the edge }\DIFaddend of the melt pond\DIFaddbegin \DIFadd{, the errors saturated to maximum -25\%. The same behaviour is observed for }\klu{} \DIFadd{but with about two times less amplitude}\DIFaddend . The mean relative errors were lower \DIFaddbegin \DIFadd{by approximately a factor of two }\DIFaddend when using \klu{} (\DIFdelbegin \DIFdel{-8}\DIFdelend \DIFaddbegin \DIFadd{-7}\DIFaddend \%) compared to \ked{} (\DIFdelbegin \DIFdel{-17}\DIFdelend \DIFaddbegin \DIFadd{-12}\DIFaddend \%). \DIFdelbegin \DIFdel{The errors of the predictions }\DIFdelend \DIFaddbegin \DIFadd{Also, the prediction errors }\DIFaddend stabilized at a shorter distance from the \DIFdelbegin \DIFdel{melt pond ice ridge }\DIFdelend \DIFaddbegin \DIFadd{centre of the melt pond }\DIFaddend when using \klu{} ($\approx$ 10 \DIFdelbegin \DIFdel{meters}\DIFdelend \DIFaddbegin \DIFadd{m}\DIFaddend ) compared with using \ked{} ($\approx$20 \DIFdelbegin \DIFdel{meters). 
Furthermore, comparing the different spatial configurations (Fig. 1) showed that the largest error occurred when the melt pond occupied 25\% of the area used to derive the reference average profile (Fig. 9}\DIFdelend \DIFaddbegin \DIFadd{m). 
}

\subsection{\DIFadd{Inelastic scattering}}

\DIFadd{Based on in situ data, our results have pointed out that }\klu{} \DIFadd{is not a good proxy for }\ked{} \DIFadd{at longer wavelengths (Supplementary Figs. 3-4) because of the effect of Raman scattering. To validate this hypothesis, we used the HydroLight (Sequoia Scientific, Inc.) radiative transfer numerical model to calculate theoretical downward irradiance and upward radiance and their associated vertical attenuation coefficients in an open water column in the presence of Raman scattering. The simulation was parameterized using IOPs measured during the field campaign (detailed information can be found in the supplementary section entitled Raman inelastic scattering). The simulation was able to reproduce the observed decoupling between }\ked{} \DIFadd{and }\klu{} \DIFadd{observed larger wavelengths $\ge$ 600 nm (Supplementary Fig. 5}\DIFaddend ).