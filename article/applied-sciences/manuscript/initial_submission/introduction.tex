\section{Introduction}

The vertical distribution of underwater light is an important driver of many aquatic processes such as primary production by phytoplankton and photochemical reactions like photodegradation of organic matter. Hence, an adequate description of the underwater light regime is mandatory to understand energy fluxes in aquatic ecosystems. In open water, when assuming an optically homogeneous water column, downward irradiance at any given wavelength follows quite well a monotonically exponential decrease with depth, which can be modeled as follows \citep{Kirk1994}:

\begin{equation}
    \edz{} = \edzero{} \times e^{-\ked(z)}
    \label{eq:edz}
\end{equation}

where \edz{} is the downward irradiance ($W~m^{-2}$) at depth $z$ (m), \edzero{} is the downward irradiance just below the surface and \ked{} is the diffuse vertical attenuation coefficient ($m^{-1}$) describing the rate at which light decreases with increasing depth. \ked{} is one of the most used apparent optical properties (AOP) of seawater and a precise estimation of this parameter is generally essential to measure or model primary production. For example, to determine primary production based on on-deck simulated incubations or photosynthetic parameters derived from photosynthesis vs. irradiance curves (P vs. E curves requires measured or estimated values of \ked{} (e.g. \citet{Morel1996}). Nowadays, \ked{} is relatively easy to estimate using commercially available radiometers. 

In the Arctic, a complex mosaic composed of ice, snow, leads, melt ponds and open water is characterizing the surface of ice-infested waters \citep{Nicolaus2013, Katlein2015, Katlein2016}. There, phytoplankton is exposed to a highly variable light regime while drifting under these features (e.g. \citet{Lange2017b}). Estimating primary production of phytoplankton under sea-ice requires an adequate approach that captures this large-area variability in the light field. In situ incubations at single locations of seawater samples inoculated with $^{14}$C or $^{13}$C are not appropriate because they reflect primary production under local light conditions, not representative or the range of irradiance experienced by drifting phytoplankton over a large area. One classical approach that is more adequate consists in conducting on-deck simulated 24h incubations of seawater samples inoculated with $^{14}$C or $^{13}$C and applying the average light attenuations at the depths of sample collection, using natural illumination and neutral filters. An alternative approach consists in calculating primary production using modeled or measured daily time series of incident irradiance, sea ice transmittance, and in-water vertical attenuation coefficients, combined with photosynthetic parameters determined on P vs. E curves measured with short ($\le$ 2h) incubations of seawater samples inoculated with $^{14}$C. Both approaches require that the vertical profile of the irradiance experienced by drifting phytoplankton be appropriately determined, which is challenging due to surface heterogeneity. Traditionally, one or very few \edz{} profiles are measured at discrete locations under sea ice \citep{Mundy2009}. Such parsimonious measurements, however, do not capture the variability induced by sea ice features. In recent studies, to better document the spatial variability of \edz{}, radiometers were attached to either remotely operated vehicles \citep{Katlein2015} or a SUIT, a net developed for deployment in ice covered waters, typically behind an icebreaker \citep{Lange2017b}. Both a ROV and the SUIT allow a better description of the light field under sea ice, which is more appropriate for determining average irradiance experienced by drifting phytoplankton. Such under-ice measurements can then be combined with \ked{} values to propagate light at depth. 

Propagating \edz{} using \ked{} values determined based on few discrete vertical profiles of \edz{} under sea-ice, a limitation that applies to any strategy for radiometer deployment, is however, very challenging because of surface heterogeneity. Indeed, under sea ice covered or not with snow,surrounded with for instance melt ponds, local \ed{} may not follow the usual monotonically exponential decrease with increasing depth (equation 1). Rather, irradiance just below sea ice few meters aside a melt pond increases with depth instead of decreasing and reaches a subsurface maximum between $\approx$5-20 meters depth \citep{Frey2011, Katlein2016, Laney2017}. Furthermore, two vertical light profiles measured few meters apart under sea ice are often very different. Hence, local measurements of light under heterogeneous sea ice do not allow an adequate description of the average light field as it would be seen by drifting phytoplankton cells at different depths. This makes estimations of primary production and the interpretation of biogeochemical data challenging in the presence of sea ice.

To fit vertical profiles of \edz{} that do not follow an exponential decay under sea ice covered with melt ponds,  \citet{Frey2011} proposed a simple geometric model (equation \ref{eq:frey2011}). 

\begin{equation}
    \edz{} = \pi \edzero{} (1 + P(N-1)\cos^2\phi)e^{-\ked(z)}
    \label{eq:frey2011}
\end{equation}

where \edzero{} is the irradiance directly below the ice/snow, $P$ the areal fraction of the ice cover, $N$ the ratio between ice and melt ponds transmittance and $\phi$ a fitting parameter defined as $\arctan(R/z)$ with $R$ the radius of the ice patch. An important drawback of this method is that additional field observations of $N$ and $P$ are required to adequately parameterize the model which makes its use more difficult. To address this concern, \citet{Laney2017} proposed a semi-empirical parameterization that includes a second exponential coefficient to equation \ref{eq:edz} to model light decrease between ice surface and ice-ocean interface.

\begin{equation}
    \edz{} = \edzero{} \times e^{-\ked(z)} - (\edzero{} - E_d(\text{NS})) \times e^{-K_{NS}(z)}
    \label{eq:laney2017}
\end{equation}

where \edzero{} is the irradiance that would be observed under homogeneous snow/ice cover, $E_d(\text{NS})$ is the irradiance under ice, $K_{NS}(z)$ describes the near-surface decrease of \edzero{}.  Both methods by \citet{Frey2011} and \citet{Laney2017} allow propagating local \edz{} vertically under specific sea ice features. Additionally, they in principle allow estimating KEd under homogeneous sea ice. What matters, when trying to determine primary production by phytoplankton that drift under sea ice and, therefore, is not static under some anecdotal sea ice feature, is the average shape of the vertical \edz{} profile, which may possibly be predictable using a large-area \meanked{} as under a wavy open-ocean surface \citep{Zaneveld2001}. 

In this study, using both in-situ data and 3D Monte-Carlo numerical simulations of radiative transfer, we show that the vertical propagation of average \edz{}, \meanedz{}, is reasonably well approximated by a single exponential decay with a so-called large-area \meanked{} under sea-ice covered with melt ponds. We further demonstrate that the large-area \meanked{} can be estimated from measurements of the vertical attenuation coefficient for upward radiance \klu{}, because the latter is supposedly less affected by local surface features of the ice cover. 