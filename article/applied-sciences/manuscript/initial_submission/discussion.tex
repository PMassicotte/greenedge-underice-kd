\section{Discussion}

In the Arctic, melt pond coverage, lead coverage and ice/snow thickness can vary highly in both time and space \citep{Landy2014, Eicken2004}. Due to this sea ice heterogeneity, local under ice measurements of downward irradiance are often characterized by subsurface light maximums (Fig. 2). To model such profiles, \citet{Laney2017} proposed a semi-empirical parameterization using two exponential terms (see equation 3). Whereas their method might provide adequate estimations of instantaneous downward attenuation coefficients at specific locations, fitting a double exponential might not be ideal because data is modelled locally and do not provide an adequate description of the average light field (\meanedz{}) as it would be seen, for example, by drifting phytoplankton cells. In such conditions it was argued that under ice, irradiance measurements should be analyzed in the context of ice and surface properties within a radius of several meters since local measurements do not reproduce the full variability of the under ice light field \citep{Katlein2015}.

Using in-situ light measurements, it was found that \ed{} and \lu{} (and therefore \ked{} and \klu{}) were highly correlated bellow 10 meters (Fig. 3, Fig. 4), even when subsurface light maxima were present (Fig. 2). One possible explanation is that a \lu{} radiometer measures scattered light originating from a larger surface area, which reduce the effect of sea ice heterogeneity. Accordingly, no subsurface light maxima were observed in the in-situ radiance profiles. This reinforce the idea that \lu{} is less influenced by sea-ice surface heterogeneity. 

Based on Monte-Carlo simulations, our results showed that the average downward light profile, \meanedz, under heterogeneous sea ice cover follows a single term exponential function, even when melt ponds occupy a large fraction of the study area (Fig. 6). This is similar to what is observed under a wavy ice-free surface \citep{Zaneveld2001}. However, estimating \edz{} for a given area is not straightforward as it requires a large number of local profiles under the sea ice. An intuitive workaround to derive attenuation coefficient is to use upward radiance which is less influenced by sea surface heterogeneity compared to downward irradiance (Fig. 2, Fig. 3, Fig. 4). Monte-Carlo simulations showed that a local estimation of \klu{} could be a good proxy for \meanked{}. Accordingly, our results showed that propagating under sea ice average irradiance (\edzero{}) using \klu{} rather than \ked{} provided better estimations of the average downward profile (Fig. 8, Fig. 9).

There are at least two main factors influencing the quality of in-situ downward measurements under heterogeneous sea ice. The first factor is the horizontal distance from the melt pond ridge. Although the relative error of propagating \edzero{} using both \ked{} and \klu{} showed the same pattern, the largest error occurred when using local estimations of \ked{} made between 1 and 10 meters outside the melt pond (Fig. 9). In contrast, in the vicinity of the melt pond, the relative errors associated to the use of \klu{} was much lower and stabilized just after approximately 5 meters. The second factor driving the relative error of local measurements is the proportion occupied by melt ponds over the area of interest (Fig. 9). Indeed, higher proportions of melt pond allows for more light to penetrate in the water column. Hence, local measurements made under surrounding ice are more likely to show subsurface light maxima (see \citet{Frey2011}). Accordingly, when melt ponds accounted for 1\% of the total area, averaged errors in \edz{} using \klu{} was 1.33\% but increased to 18\% when the melt pond occupied 25\% of the total area (Fig. 9).