\section{Results}

\subsection{Comparing in situ irradiance (\ed{}) and radiance (\lu{}) measurements}

An example showing in situ downward irradiance (\ed{}) profiles and upward radiance (\lu{}) profiles at 16 different visible wavelengths measured under ice is presented in Fig. 2. For the \ed{} profiles, subsurface light maxima at around 10 meters are clearly visible between 400 and 560 nm. These peaks are not visible in the yellow/red regions (580--700 nm). For the \lu{} profiles, no subsurface light maxima were found at any wavelength. To look closer at the shape of both \ed{} and \lu{} light profiles, data below 10 m have been normalized to the value at 10 m (Fig. 3). Below 10 m and between 400 and 580 nm, both \ed{} and \lu{} profiles follow the same pattern with increasing depth. At larger wavelengths ($\ge$ 600 nm), differences between the shapes of \ed{} and \lu{} profiles increase. A linear regression analysis between all in situ normalized \ed{} and \lu{} profiles shows that determination coefficients ($R^2$) range between 0.75 and 1 (Supplementary Fig. 3). A sharp decrease and a high variability of calculated $R^2$ occurred beyond 575 nm. This suggests a gradual decoupling between \ed{} and \lu{} profiles at larger wavelengths, likely due to the effect of inelastic scattering (mostly, Raman).

Irradiance and radiance attenuation coefficients (\ked{} and \klu) calculated on five meters depth layers are compared in Fig. 4 for all 83 profiles. In the blue/green/yellow regions (400--580 nm), the determination coefficients between \ked{} and \klu varied between 0.98 at the surface (10--15 m) and 0.64 at depth (75--80 m). For most of the surface layer, regression lines lined-up with the 1:1 lines. Slight deviations from the 1:1 lines started to appear after 60 meters where \ked{} was on average higher than \klu{}. Relationships including orange and red wavelengths are presented in Supplementary Fig. 3.

\subsection{3D Monte-Carlo numerical simulations}

\subsubsection{Simulated irradiance and radiance}

Under the melt pond, the relative density of irradiance photons was higher than that of the radiance photons (Fig. 5). Another key difference is that simulated radiance was more diffuse compared to irradiance. Over the simulated study area (50 meters radius, Fig. 1), a total of 403 irradiance profiles with and without subsurface light maxima were extracted (Fig. 6).  For these profiles, the number of "captured" photons for \ed{} varied between $4.8 \times 10^5$ and $12 \times 10^6$. Due to low scattering, the number of upward photons was much lower and ranged between 25 and 400 (Fig. 6). Averaged irradiance, (\meanedz{}), and radiance, (\meanluz{}), reference profiles showed the same pattern where the number of photons increased with increasing melt pond coverage (Fig. 6). The highest density of photons occured when the melt pond occupied 25\% of the sampling area, allowing for more light to propagate in the water column. Note that none of the averaged \meanedz{} and \meanluz{} reference profiles showed subsurface light maxima (Fig. 6).

\subsubsection{Attenuation coefficients and propagating light profiles}

Some irradiance profiles extracted outside the melt pond at distances between 5 and 15 meters of the center (to mimic local radiometric measurements) showed the same subsurface light maxima (Fig. 7) as observed on in situ profiles (Fig. 2). Beyond approximately 15 meters, subsurface light maxima disappeared and irradiance profiles followed a monotonically exponential or quasi-exponential decrease (equation 1). Note that subsurface maxima were not found on radiance profiles as with in situ data. From both irradiance and radiance profiles, \ked{} varied between 0.065 and 0.096 $m^{-1}$ and \klu{} between 0.079 and 0.1 $m^{-1}$ (Fig. 7, supplementary Fig. 6).

Propagating surface reference light (\edzero{}, surface values of the colored lines in Fig. 6) through the water column using \ked{} resulted in a greater variability compared to the profiles generated with \klu{} (Fig. 8). The relative differences between both depth-integrated (i.e. total number of photons) reference profiles and predicted profiles were used to quantify the error of using either \ked{} or \klu as a proxy to predict downward irradiance in the water column (Fig. 9).  Overall, the greatest errors on predictions reached approximately 38\% when using \ked{} at approximately 5 meters from ice ridge of the melt pond. The mean relative errors were lower when using \klu{} (-8\%) compared to \ked{} (-17\%). The errors of the predictions stabilized at a shorter distance from the melt pond ice ridge when using \klu{} ($\approx$ 10 meters) compared with using \ked{} ($\approx$ 20 meters). Furthermore, comparing the different spatial configurations (Fig. 1) showed that the largest error occurred when the melt pond occupied 25\% of the area used to derive the reference average profile (Fig. 9).