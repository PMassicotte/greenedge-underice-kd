\section{Results}

\subsection{Comparing in situ downward irradiance (\ed{}) and upward radiance (\lu{}) measurements}

An example showing in situ downward irradiance (\ed{}) profiles and upward radiance (\lu{}) profiles at 16 visible wavelengths measured under ice is presented in Fig. 3. For the \ed{} profiles, subsurface light maxima at a depth of around 10 m are clearly visible between 400 and 560 nm. These peaks are not visible in the yellow/red region (580--700 nm). For the \lu{} profiles, no subsurface light maxima were found at any wavelength. To have a closer look at the shape of both \ed{} and \lu{} light profiles, data below 10 m were normalized to the value at 10 m (Fig. 4). Below 10 m and between 400 and 580 nm, both \ed{} and \lu{} profiles presented the same shape (i.e. yield the same rate of extinction with increasing depth). At longer wavelengths ($\ge$ 600 nm), differences between the shapes of \ed{} and \lu{} profiles increased. Irradiance and radiance diffuse attenuation coefficients (\ked{} and \klu{}) calculated on layers of a 5 m depth are compared in Fig. 5 for all 83 profiles. In the blue/green/yellow regions (400--580 nm), the determination coefficients between \klu{} and \ked{} varied between 0.98 at the surface (10--15 m) and 0.64 at depth (75-80 m). For most of the surface layers, regression lines lined up with the 1:1 lines. Slight deviations from the 1:1 lines started to appear after 60 m where \ked{} was on average higher than \klu{}. The relationships including orange and red wavelengths are presented in Supplementary Fig. 3. A linear regression analysis between all in situ normalized \ed{} and \lu{} profiles showed that determination coefficients (\rsquared{}) range between 0.75 and 1 (Supplementary Fig. 4). A sharp decrease and a high variability of calculated \rsquared{} occurred beyond 575 nm. This suggests a gradual decoupling between \ed{} and \lu{} profiles at longer wavelengths, likely due to the effect of inelastic scattering (mostly, Raman). 

\subsection{3D Monte Carlo numerical simulations}

Fig. 6 shows cross-sections of the simulated downward irradiance and upward radiance. A key difference for the upcoming discussion is that the simulated upward radiance was more homogeneous compared to the simulated downward irradiance. Fig. 7 shows the reference irradiance, \edz{}, and reference radiance, \luz{}, profiles. The highest irradiance and radiance occurred when the melt pond occupied 25\% of the sampling area, allowing for more light to propagate in the water column. None of the \edz{} and \luz{} reference profiles showed subsurface light maxima. Fig. 8 shows the 50 simulated local downward irradiance and upward radiance light profiles evenly spaced by 1 m from the melt pond centre. Local downward irradiance profiles under the melt pond (0--5 m) showed a rapid decrease with increasing depth described by a monotonically exponential or quasi-exponential decrease. Local simulated downward irradiance profiles just outside the melt pond (5--10 m) were characterized with subsurface light maxima occurring at a depth of between approximately 5 and 10 m. Further away from the melt pond centre, downward irradiance profiles followed a monotonically exponential or quasi-exponential decrease. None of the simulated upward radiance light profiles presented subsurface light maxima (Fig. 8). From local simulated irradiance and radiance profiles (Fig. 8), \ked{} and \klu{} were calculated by fitting Equation 1 between 0 and 25 m. Results are presented in Fig. 9. \ked{} varied between 0.065 and 0.157 \mminus{} and \klu{} between 0.079 and 0.116 \mminus{}. These \ked{} and \klu{} were used to propagate light downward from surface reference values \edzero{}. Fig. 10 shows the profiles resulting from this operation. A greater dispersion around the reference profiles (thick black lines in Fig. 10) occurred when using \ked{} compared to the profiles generated with similarly derived \klu{} values. The relative differences between the depth-integrated values of each local profiles (coloured lines in Fig. 10) and the depth-integrated values of the reference profiles (thick black lines in Fig. 10) were used to quantify the error of using either \ked{} or \klu{} as a proxy to predict downward irradiance in the water column (Fig. 11). Below the melt pond, \ked{} overestimated the total downward irradiance by up to 40\% for the 1\% melt pond reference surface. In this region, the local $K$ coefficients are inflated. In the transition region, between 5 and 10 m from the centre of the melt pond, where subsurface maxima are observed, \ked{} underestimated the downward irradiance by up to 35\% for the 25\% melt pond reference surface. Further away from the edge of the melt pond, the errors saturated to maximum -25\%. The same behaviour is observed for \klu{} but with about two times less amplitude. The mean relative errors were lower by approximately a factor of two when using \klu{} (-7\%) compared to \ked{} (-12\%). Also, the prediction errors stabilized at a shorter distance from the centre of the melt pond when using \klu{} ($\approx$ 10 m) compared with using \ked{} ($\approx$20 m). 

\subsection{Inelastic scattering}

Based on in situ data, our results have pointed out that \klu{} is not a good proxy for \ked{} at longer wavelengths (Supplementary Figs. 3-4) because of the effect of Raman scattering. To validate this hypothesis, we used the HydroLight (Sequoia Scientific, Inc.) radiative transfer numerical model to calculate theoretical downward irradiance and upward radiance and their associated vertical attenuation coefficients in an open water column in the presence of Raman scattering. The simulation was parameterized using IOPs measured during the field campaign (detailed information can be found in the supplementary section entitled Raman inelastic scattering). The simulation was able to reproduce the observed decoupling between \ked{} and \klu{} observed larger wavelengths $\ge$ 600 nm (Supplementary Fig. 5).