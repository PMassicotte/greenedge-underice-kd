\section{Conclusions}

Our results show that under spatially heterogeneous sea ice at the surface (and for a homogeneous water column), the average irradiance profile, \meanedz{}, is well reproduced by a single exponential function. We also showed that propagating \edzero{} using \klu{} is a better choice compared to \ked{} under heterogeneous sea ice. Nowadays, radiance measurements are becoming more routinely performed during field campaigns, so we argue that one should use \klu{} when available to propagate \edzero{} through the water column under sea ice. The main difficulty remains in finding good estimates of averaged \edzero{}. In recent years, this has become easier with the development of remotely operated vehicles \citep{Katlein2015, Arndt2017, Nicolaus2013}, remote sensing techniques and drone imagery. In this study, we used a Monte Carlo approach to model an idealized surface with a single melt pond (Fig. 1, Fig. 6). Fig. 11 shows that the effect of a 5 m melt pond is minimized after approximately 20 m. Therefore, when many melt ponds are characterizing an area, if one has to perform a single profile, measuring an upward radiance profile under bare ice as far away as possible from any melt pond would minimize the error in estimating the area-averaged downward irradiance profile using \klu{}. Although not representative of a complex Arctic sea ice surface, our simple surface geometry allowed to study the transition from a high to a low transmission sea ice. Further 3D Monte Carlo work could include a more complex geometry of heterogeneous surfaces.