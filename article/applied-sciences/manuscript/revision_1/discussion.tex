\section{Discussion}

In the Arctic, melt pond coverage, lead coverage, and ice and snow thickness can vary greatly in both time and space \citep{Landy2014,Eicken2004}. Due to this sea ice heterogeneity, local under-ice measurements of downward irradiance are sometimes characterized by subsurface light maxima (Fig. 3). To model such profiles, \citet{Laney2017} proposed a semi-empirical parameterization using two exponential terms (see Equation 3). Whereas their method might provide adequate estimations of instantaneous downward diffuse attenuation coefficients at specific locations, fitting a double exponential might not be ideal because data are modelled locally and do not provide an adequate description of the average light field (\meanedz{}) as it would be seen, for example, by drifting phytoplankton cells. In such conditions, this paper argues that under-ice irradiance measurements should be analyzed in the context of ice and surface properties within a radius of several metres over the horizontal distance since local measurements cannot be used as a proxy of the average light field.

Using in situ light measurements, it was found that \ed{} and \lu{} (and therefore \ked{} and \klu{}) were highly correlated below 10 m depth (Fig. 4, Fig. 5), even when subsurface light maxima were present (Fig. 3). Furthermore, no subsurface light maxima were observed in the in situ upward radiance profiles. The reason is that a \lu{} radiometer measures upwelling photons coming from deeper depth that have undergone more scattering.  These photons thus originate from a larger surface area. This reinforces the idea that \lu{} is less influenced by sea ice surface heterogeneity. 

Based on Monte Carlo simulations, our results showed that the average downward irradiance profile, \meanedz{}, under heterogeneous sea ice cover follows a single-term exponential function, even when melt ponds occupy a large fraction of the study area (Fig. 7). This is similar to what is observed under a wavy ice-free surface \citep{Zaneveld2001}. However, estimating \meanedz{} for a given area is not straightforward, as it requires a large number of local profiles under the sea ice. An intuitive alternative to deriving the attenuation coefficient is to use upward radiance, which is less influenced by sea surface heterogeneity compared to downward irradiance (Fig. 3, Fig. 4, Fig. 5). Monte Carlo simulations showed that a local estimation of \klu{} was a good proxy for \meanked{} and that using \klu{} rather than \ked{} provided better estimations of the average downward profile by reducing the average error by approximately a factor of two (Fig. 11). 

There are at least two main factors influencing the quality of in situ downward irradiance measurements under heterogeneous sea ice. The first factor is the horizontal distance from the centre of the melt pond. Although the relative error of propagating \edzero{} using both \ked{} and \klu{} showed the same pattern, the largest error occurred when using local estimations of \ked{} directly below the melt pond and up to 10 m from the melt pond edge (Fig. 11). In contrast, the relative error associated with the use of \klu{} was much lower and stabilized just after approximately 10 m from the centre of the melt pond. The second factor driving the relative error of local measurements is the proportion occupied by melt ponds over the area of interest (Fig. 11). Indeed, higher proportions of melt pond allow for more light to penetrate in the water column. Hence, local measurements made under surrounding ice are more likely to show subsurface light maxima (see \citet{Frey2011}). Accordingly, when melt ponds accounted for 1\% of the total area, averaged error in \edz{} using \klu{} was 1.33\% but increased to 18\% when the melt pond occupied 25\% of the total area (Fig. 11).