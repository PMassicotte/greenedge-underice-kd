\section{Introduction}

The vertical distribution of underwater light is an important driver of many aquatic processes such as primary production by phytoplankton, and photochemical reactions such as the photodegradation of organic matter. Hence, an adequate description of the underwater light regime is mandatory to understand energy fluxes in aquatic ecosystems. In open water, when assuming an optically homogeneous water column, downward irradiance at any given wavelength follows, as a first approximation, quite well a monotonically exponential decrease with depth, which can be modelled as follows \citep{Kirk1994} (Equation \ref{eq:edz}):

\begin{equation}
    \edz{} = \edzero{}~e^{-\ked(z)~z}
    \label{eq:edz}
\end{equation}

\noindent where \edz{} is the downward plane irradiance (\wmsquare{}) at depth $z$ (m), \edzero{} is the downward plane irradiance (\wmsquare{}) just below the surface and $K_d(z)$ is the diffuse vertical attenuation coefficient (\mminus{}) describing the rate at which downward irradiance decreases with increasing depth. \ked{} is one of the most commonly used apparent optical properties (AOP) of seawater, and a good estimation of this parameter is important for measuring or modelling primary production. \ked{} may vary with depth because of changes in seawater inherent optical properties in the angular structure of the light field, and the effects of inelastic radiative processes such as Raman scattering by water molecules and fluorescence by phytoplankton pigments or dissolved organic matter. As \citet{Kirk1994} pointed out, for practical considerations in oceanography and limnology, the \ked{} value, even when averaged within the euphotic zone, provides a useful proxy to represent the downward irradiance attenuation in the upper water column. For example, to determine primary production based on simulated on-deck incubations or photosynthetic parameters derived from photosynthesis vs. irradiance curves (P~vs.~E~curves) requires measured or estimated values of \ked{} (e.g., \citet{Morel1996}). Nowadays, \ked{} is relatively easy to estimate using commercially available radiometers.

The ice-infested regions of the Arctic ocean are characterized by a complex mosaic made of sea ice with snow, melt ponds, ridges, and leads \citep{Nicolaus2013, Katlein2015, Katlein2016}. Phytoplankton are exposed to a highly variable light regime while drifting under these heterogeneous features (e.g., \citet{Lange2017}). Estimating primary production of phytoplankton under sea ice requires an approach that is adequate to capture this large-area variability in the light field. In situ incubations at single locations of seawater samples inoculated with $^{14}$C or $^{13}$C are not appropriate because they reflect primary production under local light conditions, which is not representative of the range of irradiance experienced by drifting phytoplankton over a large area. One classical approach that is more adequate consists in conducting on-deck simulated 24-hours incubations of seawater samples inoculated with $^{14}$C or $^{13}$C and applying the light attenuation at the depths of sample collections, using natural illumination and neutral filters. An alternative approach consists in calculating primary production using modelled or measured daily time series of incident irradiance, sea ice transmittance and in-water vertical attenuation coefficients, combined with photosynthetic parameters determined from P vs. E curves measured with short (under two hours) incubations of seawater samples inoculated with $^{14}$C. The latter two methods require that the vertical profile of the irradiance experienced by drifting phytoplankton be appropriately determined, which is challenging due to surface heterogeneity. Traditionally, one or very few \edz{} profiles are measured at discrete locations under sea ice (e.g., \citet{Mundy2009}). Such measurements, however, do not capture the variability induced by sea ice features. In recent studies, to better document the spatial variability of \edz{}, radiometers were attached to either remotely operated vehicles (ROV) \citep{Katlein2015} or a surface and under-ice trawl (SUIT), a net developed for deployment in ice-covered waters, typically behind an icebreaker \citep{Lange2017}. Both an ROV and a SUIT allow a better description of the light field right under sea ice, which is more appropriate for determining average irradiance experienced by drifting phytoplankton. Such under-ice measurements can then be combined with averaged \ked{} values to propagate light at depth.

Estimating irradiance at depth for primary production measurement or calculation using \ked{} values derived from only a few discrete vertical profiles of \edz{} under heterogenous sea ice is problematic whatever the platform for radiometer deployment. Let us consider that phytoplankton, by continuously drifting horizontally relative to sea ice, are exposed to fluctuations in irradiance due to surface heterogeneity, and that the relevant light metrics for primary production in such conditions is irradiance at any depth averaged over some horizontal area. When measuring an irradiance profile at one given location under sea ice, as the depth of the upward-looking detector increases, light from a larger area on the underside of the ice enters the detector field of view. In other words, the detector "sees" different things at different depths. One consequence is that \edz{} measured that way may not follow the usual monotonically exponential decrease with increasing depth (Equation 1). For example, irradiance profiles measured beneath low-transmission sea ice (e.g., white ice) relative to surrounding areas showing melt ponds, show subsurface light maxima. The literature reports subsurface maxima varying between 5 m and 15 m in depth \citep{Frey2011, Katlein2016, Laney2017}. Conversely, it is also important to note that \ked{} estimations are biased when profiles are measured beneath an area of high transmission (e.g., a melt pond) relative to surrounding areas \citep{Katlein2016}. Indeed, with depth, light decreases more quickly than what would be expected from the inherent optical properties (IOPs) of the water column. In the field, this situation is more difficult to identify compared to profiles showing subsurface maxima because the former measurements may appear to follow a single exponential decrease but would not produce a diffuse attenuation coefficient that adequately describes the water mass. So, two vertical light profiles measured a few metres apart under sea ice are often very different. More importantly, local measurements of light under heterogeneous sea ice do not provide an adequate description of the average light field as it would be seen by drifting phytoplankton cells at different depths. This makes estimations of primary production and the interpretation of biogeochemical data challenging in the presence of sea ice.

To fit vertical profiles of \edz{} under bare ice that do not follow an exponential decay under sea ice covered with melt ponds, \citet{Frey2011} proposes a simple geometric model (Equation \ref{eq:frey2011}). 

\begin{equation}
    \edz{} = \pi \edzero{} (1 + P(N-1)\cos^2\phi)e^{-\ked(z)~z}
    \label{eq:frey2011}
\end{equation}

\noindent where \edzero{} is the irradiance directly below the ice/snow, $P$ the areal fraction of the ice cover, $N$ the ratio between ice and melt ponds transmittance and $\phi$ a fitting parameter defined as $\arctan(R/z)$ with $R$ the radius of the ice patch and $z$ the depth. A major drawback of this method is that additional field observations of $N$ and $P$ are required to adequately parameterize the model, which makes its use more difficult. To address this concern (among others), \citet{Laney2017} proposed a semi-empirical parameterization that includes a second exponential coefficient in Equation \ref{eq:edz} to model light decrease at the interface between the ice and ocean water at the bottom of the ice layer (Equation \ref{eq:laney2017}):

\begin{equation}
    \edz{} = \edzero{}e^{-\ked(z)~z} - (\edzero{} - E_d(\text{NS}))~e^{-K_{NS}(z)~z}
    \label{eq:laney2017}
\end{equation}

\noindent where \edzero{} is the irradiance that would be observed under homogeneous snow or ice cover, $E_d(\text{NS})$ is the irradiance under ice, and $K_{NS}(z)$ describes the decrease of \edzero{} just under the ice layer. Both the methods by \citet{Frey2011} and \citet{Laney2017} make it possible to propagate local \edz{} vertically under low transmission ice. However, these methods cannot identify and correct for inflated \ked{} when profiles are measured beneath an area of high transmission relative to surrounding areas. Additionally, when trying to determine primary production by phytoplankton that drift under sea ice and therefore are not static under sea ice features, what matters is the average shape of the vertical \edz{} profile, which may possibly be predictable using a large-area \meanked{} as under a wavy open ocean surface \citep{Zaneveld2001}. 

In this study, using both in situ data and 3D Monte Carlo numerical simulations of radiative transfer, we show that the vertical propagation of average \edz{}, \meanedz{}, is reasonably well approximated by a single exponential decay with a so-called large area \ked{}, \meanked{}, under sea ice covered in melt ponds. We further demonstrate that \meanked{} can be estimated from the vertical attenuation coefficient for upward radiance (\klu{}) because the latter is apparently less affected by local surface features of the ice cover. We implicitly assume that primary production can be adequately modeled using \meanedz{}, and we conclude that \klu{} is an appropriate AOP for predicting the vertical variations in \meanedz{} under sea ice.