\section{Conclusions}

Our results show that under spatially heterogeneous sea ice surface, the average light profile, \meanedz{}, is well reproduced by a single exponential function. We also showed that propagating \edzero{} using \klu{} is better a choice compared to \ked{} under heterogeneous sea ice. Nowadays, radiance measurements are becoming more routinely measured during field campaigns of ecological studies, so we argue that one should use \klu{} when available to propagate \edzero{}. The main difficulty remains at finding good estimates of \edzero{}. In the last years, this became easier with the development of remotely operated vehicles \citep{Katlein2015, Arndt2017, Nicolaus2013}, remote sensing techniques and drone imagery. In this study we used a Monte-Carlo approach to model an idealized surface with a single melt pond (Fig. 4). However, in the Arctic, a complex mosaic composed of ice, snow, leads, melt ponds and open water is characterizing the landscape. Hence, further work using 3D radiative transfer models are needed to fully understand light distribution under spatially heterogeneous surface.