\documentclass[12pt,a4paper]{scrartcl}
    \usepackage[utf8]{inputenc}
    \usepackage{amsmath}
    \usepackage{amsfonts}
    \usepackage{amssymb}
    \usepackage{graphicx}

    \usepackage[bottom = 1in, left = 0.5in, right = 0.5in, top = 1in]{geometry}

    \usepackage[english]{babel}
    \usepackage[autostyle]{csquotes}
    \usepackage{mathptmx}

    \usepackage[labelfont=bf]{caption}

    \usepackage[default, scale=0.95]{opensans}

    \usepackage[T1]{fontenc}

    \usepackage{fixltx2e}

    \title{Figures}
    \date{}

    \begin{document}
    \maketitle

     \begin{figure}[ht]
         \centering
         \includegraphics[scale = 1]{../../../graphs/fig1.pdf}
         \caption{Example of downwelling irradiance (Ed) and upwelling radiance (Lu) light profiles taken under-ice (between 400 and 700 nm). Note the subsurface light maximum for the irradiance profiles that are not present in the radiance profiles.}
     \end{figure}

    \clearpage
    \newpage

    \begin{figure}[ht]
        \centering
        \includegraphics[scale = 1]{../../../graphs/fig2.pdf}
        \caption{Comparison of downwelling irradiance (Ed) and upwelling radiance (Lu) of one light profile taken under-ice. Profiles were normalized to light at 10 meters (under subsurface light maximum) in order to emphasize the similar shape between Ed and Lu.}
    \end{figure}

    \clearpage
    \newpage

    \begin{figure}[ht]
        \centering
        \includegraphics[scale = 1]{../../../graphs/fig3.pdf}
        \caption{Averaged determination coefficient (A),  slope (B) and intercept (C) of the regressions between downwelling irradiance (Ed) and upwelling radiance (Lu) light profiles taken under-ice (between 400 and 700 nm). Profiles were normalized to light at 10 meters (under subsurface light maximum) in order to emphasize the similar shape between Ed and Lu. Shaded area represent the standard deviation.}
    \end{figure}

    \clearpage
    \newpage

    \begin{figure}[ht]
        \centering
        \includegraphics[scale = 1]{../../../graphs/fig4.pdf}
        \caption{Scatter plots showing the relationships between downwelling irradiance (Ed) and upwelling radiance (Lu). Blue lines and shaded areas represent respectively the regressions lines and the confidence intervals of the fitted linear models. Dotted lines are the 1:1 lines.}
    \end{figure}

    \clearpage
    \newpage

    \begin{figure}[ht]
        \centering
        %\includegraphics[scale = 1]{../../../graphs/fig4.pdf}
        \caption{Scatter plots comparing Ked derived from Klu and 3D simulations.}
    \end{figure}

    \end{document}
