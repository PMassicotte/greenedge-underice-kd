\documentclass[12pt,a4paper]{scrartcl}
    \usepackage[utf8]{inputenc}
    \usepackage{amsmath}
    \usepackage{amsfonts}
    \usepackage{amssymb}
    \usepackage{graphicx}

    \usepackage[bottom = 1in, left = 0.5in, right = 0.5in, top = 1in]{geometry}

    \usepackage[english]{babel}
    \usepackage[autostyle]{csquotes}
    \usepackage{mathptmx}

    \usepackage[labelfont=bf]{caption}

    \usepackage[default, scale=0.95]{opensans}

    \usepackage[T1]{fontenc}

    \usepackage{fixltx2e}

    \addto\captionsenglish{\renewcommand{\figurename}{Supplementary Fig.}}
    \addto\captionsenglish{\renewcommand{\tablename}{Supplementary Table}}

    \title{Appendix}
    \date{}

    \begin{document}
    \maketitle

     \begin{figure}[h]
         \centering
         \includegraphics[scale = 0.75]{../../../graphs/supp_fig_1.pdf}
         \caption{Study site.}
     \end{figure}

    \clearpage
    \newpage

    \begin{figure}[h]
        \centering
        \includegraphics[scale = 1]{../../../graphs/supp_fig_2.pdf}
        \caption{Scatter plot showing the relationship between downwelling irradiance (Ed) and upwelling radiance (Lu) ($n$ = 7471).}
    \end{figure}

    \clearpage
    \newpage

    \begin{figure}[h]
        \centering
        \includegraphics[scale = 1]{../../../graphs/supp_fig_3.pdf}
        \caption{Scatter plots showing the relationships between downwelling irradiance (Ed) and upwelling radiance (Lu) including all the 15 wavelengths. Blue lines and shaded areas represent respectively the regressions lines and the confidence intervals of the fitted linear models. Dotted lines are the 1:1 lines. See also Fig. 3.}
    \end{figure}

    \end{document}
